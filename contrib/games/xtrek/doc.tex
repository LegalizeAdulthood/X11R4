% doc.tex
% this contains some doc about xtrek
% much of it is taken directly from the original xtrek doc

\title{Xtrek version 5.4}

\author{Daniel Lovinger dl2n+@andrew.cmu.edu}
\author{Jon Bennett jcrb@cs.cmu.edu}
\author{Mike Bolotski mikeb@salmon.ee.ubc.ca}
\author{David Gagne daveg@salmon.ee.ubc.ca}

\date{September 29, 1989}

\parindent 0in
\parskip 0.10in 
\textwidth 6.5in
\oddsidemargin 0in
\evensidemargin 0in
\textheight 8.175in
\topmargin -0.25in

\begin{document}

\maketitle

\tableofcontents \newpage

\section{Introduction}

Xtrek is one of the all time classic games for X Windows. This release of the game at
version 5.4 represents almost five straight months of work by various people at 
Carnegie Mellon University and the University of British Columbia, and in our humble
opinion constitutes a major improvement over the previous version 4.0. For those familiar
with version 4.0, much of the game remains on the surface the same : you still fly a ship
'round the galaxy blowing away your friends and conquering planets. However, where the old
xtrek was cast in stone (or executable), this version is configurable in almost every respect
from the power of your torpedoes to the number of configuration of the galaxy. A whole new
universe is only a few hours of configuration file hacking away ...

The basic object of the game is to conquer sixteen (by default - as with almost everything, 
this is configurable) planets for your empire. Of course, everyone else is also, and that is where
all of the fun comes in.

\section{Getting Started}

Xtrek is split up into two seperate programs, xtrek and xtrekd. Xtrekd is the main server process for the game
and accepts as a command line argument the name of a configuration file for the game. For instance

xtrekd cmu.config

will start a game with the contents of the cmu.config file. The daemon first tries to open the config file
as an absolute pathname, and then attempts to load the config file from the xtrek library directory (specified
on compile time). If no config file is specified, ``default.config'' is read from the library. The daemon will
print out status information during the course of the game, including the number of players each time a death
or entrance occurs, and notifies you whenever a robot enters for a given empire.

\subsection{Entering the Game}

Once xtrekd is running, players can connect to it by running xtrek with the machine name the daemon is running
on as the argument on the command line. For instance, assume that a player has a running xtrekd on ligonier.andrew.cmu.edu.
Then

xtrek ligonier.andrew.cmu.edu

would send a request to the daemon to open an xtrek window on the local machine. Various errors can occur at this point,
most of which have to do with access protections on you X server - you must first xhost to the machine running the
daemon.

\subsection{Screen Layout}

\subsection{Command Reference}

\subsection{Shooting}

\section{Ship Details}

\subsection{Cooling}

Firing weapons and running your engines generates heat which must be dissapated through
your ship's interlinked cooling system. As your systems heat up toward their maximum of
100% loaded, they try to dump their excess heat off into the opposite cooling system. This 
can be extremely useful when you are desperately trying to escape an enemy trap, and need
just that extra boost to get you into home space. Each weapon shot 'costs' given amount of
heat which is based on the power of the weapon. Warp drive on the other hand costs an amount
that increases as you increase your warp factor. With all of the above interactions taken into
account, it is usually possible for a normal ship to pull about warp five before starting
to accumulate heat.

In addition to these standard sources of heating if you are playing a configuration with
the teleport option, teleporting costs a large amount.

The cooling systems can only operate at peak efficiency with ship's shields down since the
entire purpose of the shield system is to repel energy. With shields up the cooling system
will still operate, but at a lower dissapation rate. Similarly, when the ship is cloaked the
cooling systems operate at a scaled back rate.

If the ships cooling system exceed 100% capacity, there is a growing probability during each
time slice that the system being cooled will shut down to protect itself. The capacity of the
cooling system must approach 0% for the system to be reactivated. When a system 'temps', a flag
is triggered on the status line, W for weapns, E for engines.

\subsection{Warp Factor}






\subsection{Status Area}

The status area contains your vessel's 'vital signs.'
If you are locked onto another ship, in addition to your own you
will get the status of the other vessel on the second line, minus
the \bold{Flags} field. There are eleven seperate sections to the
full status line :

\begin{itemize}
\item Flags
	This field contains information about ship activities.  There are twelve status characters
	that can be activated in this section (in order, left to right):

	\begin{itemize}

	\item \typewriter{S   }-- ship has shields up
	\item \typewriter{GYR }-- green, yellow, or red alert
	\item \typewriter{L   }-- ship has sensor or planet lock activated
	\item \typewriter{R   }-- ship is under repair}
	\item \typewriter{B   }-- ship is bombing a planet}
	\item \typewriter{O   }-- ship is orbiting a planet}
	\item \typewriter{C   }-- ship is cloaked}
	\item \typewriter{W   }-- ship's weapons are over operating temperature}
	\item \typewriter{E   }-- ship's engines are over operating temperature}
	\item \typewriter{u   }-- ship is beaming armies up from planet surface}
	\item \typewriter{d   }-- ship is beaming armies down to planet surface}
	\item \typewriter{P   }-- ship is allowing co-pilots}

	\end{itemize}

\item Warp -- shows the ship's current warp speed in x.x format

\item Damage -- shows the ship's current hull damage

\item Shield -- shows the number of damage points your shield
	generators can absorb

\item Torpedoes -- shows the number of torpedoes the ship has flying

\item Kills -- shows the number of kills the ship has prosecuted in x.xx format

\item Armies -- shows the number of armies currently onboard

\item Fuel -- shows the current amount of fuel onboard

\item Weapon Temp -- weapon temperature as a percentage of maxmimum

\item Engine Temp -- engine temperature as a percentage of maxmimum

\item Failed Systems
	This field contains info about which of the ship's systems are currently
	non-operational due to excessive damage.  If a letter is lit, the system
	is down (in order, left to right)

	\begin{itemize}

	\item \typewriter{C  }-- cloaking device
	\item \typewriter{L  }-- long range sensors
	\item \typewriter{P  }-- phaser banks
	\item \typewriter{S  }-- short range sensors 
	\item \typewriter{T  }-- torpedoes
	\item \typewriter{c  }-- cooling system
	\item \typewriter{l  }-- sensors (locking ablility)
	\item \typewriter{s  }-- shields
	\item \typewriter{t  }-- transporters

	\end{itemize}
\end{itemize}

\subsubsection{Armies}
\subsection{Ship Subsystems}
\subsubsection{Photon Torpedoes}
\subsubsection{Phasers}
\subsubsection{Shields}
\subsubsection{Engines}
\subsubsection{Cloak}
Note that engines and weapons cool 25\% slower when
cloak is up.

\subsubsection{Mines}
\subsubsection{Sensors}
\subsubsection{Transporter}

\section{Planets}

\subsection{Conquering Planets}

Planets are conquered by first bombing them to reduce the number of
armies to a manageable level and then beaming down your own armies.  A
ship must be in orbit to bomb a planet.  A side effect of this
requirement is that shields are dropped and planetary defensive fire
damages the hull directly.  Planetary fire causes damage proportional
to the number of defending armies.  The actual formula is
$(\mbox{armies}/ 10) + 2$ damage points twice per second.

\subsection{Fuel and Repair}
Fuel replenishes at twice the normal rate when orbiting a planet.  Certain
planets are marked as {\em fuel} sources;  orbiting these planets doubles
the rate again.

\section{Default Scenario}

%this is taken from Dan's informal description waay back in July 

Klingon : heavy phasers, low torps. normal engines, but they
can't slow down very fast, so turning is slower than normal. low fuel
gain-back, but they make up for that by toasting anything that gets
near with the phasers. best cloaking costwise ...

Romulan : nickname 'garbage scow', pretty accurate. slowest
turn rate in the game. heavy torps that will rail an opponent that
gets caught in a stream. best shields, and in terms of how much energy
they chew, best cloaking since they also have the best energy regen
rate. hull strength also highest.

Orion : tend to spit torps, which are generally weak but very
cheap and fast to fire. best acel/decel in the game, and turn on a
dime. hull pts and shield are the lowest, so they need those engines
of theirs often. regen is also low, so they need accuracy on those
torp shots. phasers are also weak, and don't have a very good range.
mostly useful for annoyance shots and close in fighting.

Federation : this is basically the standard xtrek ship of old
(we had to keep one of them around :-). All around average ship. High
cloak cost, but also have a decent regen rate. Best repair rate
(Scotty factor :-), and their weapons are both pretty potent. Can get
out-paced if they don't watch out, though - especially by Orions.

\section{Scenario Customization}
\subsection{Configuration File Syntax}

All parameters are integers unless specified otherwise.

\subsection{Ship Parameters}
\begin{itemize}
\item[turns]  Ship maneuverability (turn radius)
\item[torp-damage]  Torpedo damage at zero radius
\item[torp-blast-range]  Torpedo blast radius - range at which damage = 0
\item[mine-damage] Mine damage at zero radius
\item[mine-blast-range]  Mine blast radius - range at which damage = 0
\item[phaser-damage] Phaser damage at zero radius
\item[phaser-range] Phaser range - range at which damage = 0
\item[ph-pulses]  Number of time slices that a phaser will lock on a target
\item[torp-speed]  Torpedo speed
\item[max-speed] Maximum ship speed
\item[shield-repair]  Shield repair rate
\item[hull-repair]  Damage repair rate
\item[max-fuel]  Maximum fuel capacity
\item[torp-cost] Torpedo fuel cost (unused) .  Note that weapons costs are
calculated from the damage caused by each weapon
\item[mine-cost] Mine fuel cost (unused)
\item[phaser-cost] Phaser fuel cost (unused)
\item[det-cost] Detonate other torps fuel cost
\item[warp-cost] Warp engine fuel cost
\item[cloak-cost] Cloak fuel cost
\item[recharge]  Fuel recharge rate
\item[accint]  Acceleration rate
\item[decint]  Deceleration rate
\item[max-armies]  Maximum army capacity
\item[weapon-cool]  Weapon cooling rate
\item[engine-cool]  Engine cooling rate
\item[max-damage] Maximum structural damage sustainable before destruction
\item[shield-max]  Maximum shield damage absorption
\item[teleport-heat]  Teleport engine heat cost
\item[teleport-cost]  Teleport fuel cost
\item[teleport-range]  Teleport radius
\item[turbo-speed]  Turbo warp speed (unused)
\item[turbo-time]  Turbo warp duration (unused)
\item[reload] time between torp salvos
\item[burst] number of shots in a torp salvo
\item[phaser-fail]  Probability of phaser failure.  Similarly for 
{\tt torp-fail}, {\tt trans-fail}, {\tt shield-fail}, 
 {\tt cloak-fail}, {\tt lrsensor-fail}, {\tt srsensor-fail}, {\tt lock-fail},
and {\tt cooling-fail}.
\end{description}

\subsection{Robot Parameters}
\begin{description}
\item[hscruise] speed robots go when going to assist another robot.
\item[cruise] speed robots patrol at.
\item[battle] speed robots go in battle.
\item[flee] speed robots run away at.
\item[cloaked] speed robots go when cloaked.
\item[refresh] speed robots go when they want to cool down.
\item[engage] distance at which a robot starts firing at a target.
\item[disengage] distance at which a robot stops firing at a target.
\item[shotdamage] damage a robot's salvo does.
\item[circledist] distance at which a robot would like to stand off it's
target.
\item[sneaky] chance that, on creation, a robot will elect to make cloaked
attacks.
\item[max-cloak] distance at which a robot, making a sneaky attack, will
cloak.
\item[min-cloak] distance at which a robot, making a sneaky attack, will uncloak.
\end{description}

\subsection{Empire Parameters}
\begin{description}
\item[empire] String. Empire's name.
\item[icon] String parameter specifies bitmap file which is in the defined LIBDIR.  Bitmap is 
automatically rotated through all 16 orientations.  The bitmap should be specified
with front pointing upwards, and some slight aliasing will result.
\item[robotname]  String parameter specifies robot name.
\end{description}

\subsection{Planet Parameters}
\begin{description}
\item[planet] String. Planet's name.
\item[empire] String. Planet's empire.
\item[home]  Boolean flag.  Specifies that planet is the home planet of the
empire.
\item[fuel] Boolean flag.  Specifies that planet is a fuel source; fuel is
replenished at twice normal planetary rates when planet is orbited. 
\item[repair] Boolean flag.  Specifies that planet is a repair yard; damage is
repaired at twice normal planetary rates when planet is orbited. 
\item[x]  Specifies X coordinate of planet.  Similarly for {\tt y}.
\item[armies] Specifies number of armies on planet at start of game.
Note that a large number of  armies makes the planet an effective obstacle;
any ship passing through such a planet will be destroyed.
\end{description}

\subsection{Global Parameters}
\begin{description}
\item[robot-giveup-time] If after this many seconds there are no players
in the game the robots quit. Defaults to 10.
\item[death-time] Player is dead for this number of seconds.  Defaults to 4. 
\item[torp-life-min]   Torpedoes live for at least this number of seconds.  
Defaults to 7.  Similarly for {\tt mine-life-min} (default is 60).
\item[torp-life-var]  Torpedoes may live for at least this number of seconds 
in addition to the minimum lifetime.  Defaults to 2. Similarly for 
{\tt mine-life-var} (default is 120).
\item[player-explode-time] Player explodes for this number of seconds.
\item[weapon-lock-min] Overheated weapons are unusable for at least this
number of seconds  up   to an additional {\tt weapon-lock-var} seconds.
Defaults are 10 and 15. Similarly for {\tt engine-lock-min} and 
{\tt engine-lock-var}.  Defaults are  identical to those of weapons.
\item[self-destruct-time]  Self-destruct countdown time.
\item[robot-giveup-time] Robots will give up within this number of
seconds if no combat is initiated.
\item[orbit-speed]   Maximum warp allowed while entering orbit. 
\item[det-dist]  Effective range of the counter-torpedo batteries. 
\item[orbit-dist]  Range at which a ship can enter orbit. Defaults to 900.
\item[planet-fire-dist]  Range at which a planet can damage a ship.
Defaults to 1500.
\item[phaser-hit-angle]  Number of degrees that a phaser will hit.
\item[phaser-hit-langle]  Number of degrees that a locked phaser will hit.
\item[auto-quit]  Auto-logout countdown time.
\item[cool-penalty]   Floating point number.  Specifies the increase
in weapon and engine cooling rate when both shields and cloak are inactive.
Defaults to 1.25.
\item[fast-destruct]  Boolean flag.  If set, self-destruct time does not
increase with alert level.   Default is to double self-destruct time for
yellow alert, and double again for red alert.
\item[mine-detonate]  Boolean flag.   If set, mines detonate when
destroyed.  This feature is potentially very nasty.  It also allows instant
self-destruction, cheating players out of kills.
\end{description}


\section{Robots}
\subsection{The Killer Robots from Hell}

The robots strategy is based on the following variables, and the code that
interprets them.

\subsection{Speeds}
\begin{description}
\end{description}


\section{Hints}
\subsection{X Customization}
\subsection{Credits, Bug Reports, etc}
\subsection{Key Table}

\end{document}
