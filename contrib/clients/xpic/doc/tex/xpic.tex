% Document type: LaTeX
%
% $Header: xpic.tex,v 1.2 88/11/04 23:01:46 moraes Exp $
%
%  Copyright 1988, Mark Moraes
%  May be freely used for non-commercial purposes provided acknowledgement
%  to the source is given.
%
%%%%%%%%%%%%%%%%%%%%%%%%%%%%%%%%%%%%%%%%%%%%%%%%%%%%%%%%%%%%%%%%%%%%%%%%%%%%%%
% Specifies the document style.
\documentstyle[twoside,fullpage,notes]{article}

\def\Ps{Post\-Script}
\newcommand{\postscript}{Post\-Script\ }
\newcommand{\psfigtex}{Psfig/\TeX\ }
\newcommand{\tex}{\TeX\ }
\newcommand{\nodeeqn}{$ {\displaystyle y = f \left( \sum_{i=1}^{n-1}
w_{i} x_{i} - \theta \right)} $}

% From the TeXbook, Appendix E, pg 419
\def\dbend{{\manual\char127}} % "dangerous bend" sign
\def\d@nger{\medbreak\begingroup\clubpenalty=10000
  \def\par{\endgraf\endgroup\mdebreak} \noindent\hang\hangafter=-1
  \hbox to0pt{\hskip-\hangindent\dbend\hfill}\ninepoint}
\outer\def\danger{\d@nger}

\input{psfig}
\title{Including {\tt xpic} figures in \LaTeX}
\author{Mark Moraes\\
moraes@csri.toronto.edu}
\date{November 5, 1988}    % Deleting this command produces today's date.

\begin{document}           % End of preamble and beginning of text.
\maketitle                 % Produces the title.

\section{Introduction}

{\tt Xpic} is a program for drawing figures under {\em the X Window
system}. It produces a file in a special format. Various programs
exist to translate from this format to formats which may be included
in \tex documents to eliminate the pains of manual paste-up.
This document describes two ways to do just that.
\begin{figure}[h]
\centerline{\psfig{figure=texwarning.ps,height=1.5in}}
\caption{A simple example. Don't take the warning seriously!}
\label{fig:texwarning}
\end{figure}
\section{\psfigtex} The first method is to convert the {\tt xpic}
figure to \postscript and include the \postscript in the document.
The conversion is done with the {\tt x2ps} program, which can be run
as follows
\begin{quote}
\tt
x2ps file.x > file.ps
\end{quote}
where {\tt file.x} is the {\tt xpic} figure and {\tt file.ps} is the
resulting \postscript figure.

\psfigtex is a macro package for \tex that facilitates the inclusion of
arbitrary \postscript figures into \tex documents. Figures are
automatically scaled and positioned on the page, and the proper amount
of space is reserved. The only restriction is that the \postscript
figures must be {\em well behaved} and
must adhere to the bounding box comment convention.\footnote[1]{
See `Appendix J: \Ps\ File Structuring Conventions' in
{\it The \Ps\ Language Reference Manual}}
{\tt xpic} follows
this convention, so inclusion of such figures in \psfigtex is easy.

To include a \Ps\ figure with {\tt psfig}, first load the {\tt psfig}
macros at the beginning of your document with
\begin{quote}
{\tt\verb+\input{psfig}+}
\end{quote}
\postscript files may then be included in the document using the commands
\begin{quote}
\tt
{\tt\verb+\+psfig\{figure={\it file.ps}\}}
\end{quote}
Psfig will automatically position the figure at the current point on the page, 
and reserve the proper amount of space in \TeX\ so that it doesn't conflict
with any other objects on the page.

For example, we included the {\tt xpic} figure in the file {\tt
texwarning.x} as figure \ref{fig:texwarning} by first converting it to the
\postscript file {\tt texwarning.ps} using the {\tt x2ps} command as
follows ---
\begin{quote}
\begin{verbatim}
x2ps texwarning.x > texwarning.ps
\end{verbatim}
\end{quote}
We could then use the commands
\begin{quote}
\tt
\begin{verbatim}
\begin{figure}[h]
\centerline{\psfig{figure=texwarning.ps}}
\caption{A simple example. Don't take the warning seriously!}
\label{fig:texwarning}
\end{figure}
\end{verbatim}
\end{quote} 
to include it as a centered paragraph.  Since no mention
of size was made in the above example, \psfigtex draws the figure at
its natural size (as if it was printed directly on a \Ps\ printer.) The
figure's natural size is several inches high, which is a little large;
Figure \ref{fig:texwarning} in the introduction was produced with the actual
{\tt psfig} invocation changed to:
\begin{quote}
\tt\verb+\+centerline\{\verb+\+psfig\{figure=texwarning.ps,height=1.5in\}\}
\end{quote} 
The {\tt height} option specifies how tall the figure
should be on the page. Since no {\tt width} was specified, the figure
was scaled equally in both dimensions. By listing both a {\tt height}
and a {\tt width}, figures can be scaled disproportionately, with
interesting results.  For example:

\begin{figure}[h]
\centerline{\hbox{
\psfig{figure=funstuff.ps,height=.8in,width=.15in}
\psfig{figure=funstuff.ps,height=.8in,width=.35in}
\psfig{figure=funstuff.ps,height=.8in}
\psfig{figure=funstuff.ps,height=.8in,width=1.2in}
\psfig{figure=funstuff.ps,height=.8in,width=1.5in}
}}
\caption{Abstract art. Untitled}
\label{fig:funstuff}
\end{figure}

was produced with:

\begin{quote}
\tt
\begin{verbatim}
\begin{figure}[h]
\centerline{\hbox{
\psfig{figure=funstuff.ps,height=.8in,width=.15in}
\psfig{figure=funstuff.ps,height=.8in,width=.35in}
\psfig{figure=funstuff.ps,height=.8in}
\psfig{figure=funstuff.ps,height=.8in,width=1.2in}
\psfig{figure=funstuff.ps,height=.8in,width=1.5in}
}}
\caption{Abstract art. Untitled}
\label{fig:funstuff}
\end{figure}
\end{verbatim}
\end{quote}

\section{\tt Tpic}

Even though \psfigtex is a powerful and easy-to-use method of
including \postscript figures in a document, it has two drawbacks. The
first is that all output devices used for \tex do not have
\Ps. The other problem is that strings put in a
\postscript figure are not processed by \TeX, so if you want to put
equations in, you have to overlay them on top of the picture.
(You can measure distances with a ruler and use the {\tt\verb+\picture+} 
environment to do this, using {\tt \verb+\put+} commands to place the 
equations.)

A way of getting around this is to use a picture drawing preprocessor that
passes the text through \TeX. One such preprocessor is {\tt tpic}.\footnote{
{\tt Tpic} is a port of {\tt troff} 's {\tt pic} preprocessor.}

To convert {\tt xpic} output to {\tt tpic} format, use the {\tt x2tpic}
command, then run {\tt tpic} to convert this to \TeX. You can then include
this \tex file produced in a document. For instance we include the
figure
{\tt node.x}, with the following commands:\footnote{
The command {\tt x2tex} does the same thing as running the following two
commands. It is sometimes more flexible to use the two commands separately.}
\begin{quote}
\tt
x2tpic node.x > node.pic \\
tpic node.pic 
\end{quote}
We now have a file called {\tt node.tex} with a picture called
{\tt\verb+\graph+} in it --- the following \LaTeX\
commands are used to include it:
\begin{quote}
\tt
\begin{verbatim}
\begin{figure} [h]
\input{node}		% This reads in the picture definition
\centerline{\box\graph}		% This positions the picture and draws it
\caption{Computational element or node which forms a weighting 
sum of $ n $ inputs}
\label{fig:node}
\end{figure}
\end{verbatim}
\end{quote}
This produces the picture in figure \ref{fig:node}.
\begin{figure} [h]
\input{node}
\centerline{\box\graph}
\caption{Computational element or node which forms a weighting 
sum of $ n $ inputs}
\label{fig:node}
\end{figure}
The figure is little large, so we generate a smaller version asking
{\tt x2tpic} to scale it down for us with:
\begin{quote}
\tt
x2tpic -s 0.6 node.x > shrunknode.pic \\
tpic shrunknode.pic 
\end{quote}
and include it as figure \ref{fig:shrunknode}.
\begin{figure}[h]
\input{shrunknode}
\centerline{\box\graph}
\caption{Computational element or node which forms a weighting 
sum of $ n $ inputs}
\label{fig:shrunknode}
\end{figure}
\par
{\small {\em Hint} --- The text for the equation
in the box above is 
\begin{verbatim}
$ {\displaystyle y = f \left( \sum_{i=1}^{n-1} w_{i} x_{i} - \theta \right)} $
\end{verbatim}
which is quite long, and appears literally in the {\tt xpic} figure
since {\tt xpic} does not process equations on the screen.  This looks
slightly messy onscreen and overflows the edges of the picture --- a
better way to do it is to define a command to generate the equation,
and use that in the figure. Then in the document, you could say
something like
\begin{verbatim}
\newcommand{\nodeeqn}{ $ {\displaystyle y = f \left( \sum_{i=1}^{n-1} 
w_{i} x_{i} - \theta \right)} $ }
\end{verbatim}
and put \verb+\+nodeeqn in the picture instead of the long equation.}
\par
{\small {\em Hint} --- {\tt xpic}'s text alignment options are
important if you want to get good results. For instance, in the figure
above, most text, including the equation in the box uses the default {\tt
Centered, Middle} attributes, which mean that the single {\em control
point} whose location is guaranteed is the centre of the text. For the
labels $ x_{0}, x_{1}, x_{n-1} $ along the left edge, we use the {\tt
Right Justified, Middle} attributes so that we ensure that they are all
aligned along the same right edge. Remember that {\tt xpic} screen
fonts are only approximations of the actual fonts used by \TeX, and
aren't too accurate.}
\section{Further Reading}
For more information on \psfigtex --- see the {\em
\psfigtex 1.2 Users Guide} by Trevor Darrell ({\em
trevor@grasp.cis.upenn.edu}). Some of
the examples used in this note are based on that document.

{\tt Pic} is a powerful language for describing pictures --- {\tt xpic}
uses only a small subset. For more information on the {\tt pic}language, see {\em PIC --- A Graphics Language for Typesetting} by
Brian W. Kernighan. {\tt Pic} is really a preprocessor for the {\tt
troff} typesetting program, but the language is quite independent of
the underlying typesetter. A few differences are listed in the {\tt
tpic} manual page.

{\tt Xpic, x2ps} and {\tt x2tpic} are described in {\em Using xpic} by
Mark Moraes. {\em moraes@csri.toronto.edu}. These programs also have
manual pages.

\section{Availability}
\psfigtex is available by anonymous ftp from {\em linc.cis.upenn.edu}
(Internet host number 128.91.2.8), in the tar file
{\tt ~ftp/dist/psfig/tex.tar.Z}. There exists a corresponding {\tt troff}
version.

{\tt Pic} is part of the {\em Documenter's Workbench} sold by AT\&T
(and many vendors of Unix.\footnote{Unix is a trademark of AT\&T}). It
is also supplied with research editions of Unix. (eg) Eighth Edition.

{\tt Tpic} is a port of {\tt pic}. Therefore, you need a Unix source
license to get it. Contact Tim Morgan {\em morgan@rome.ics.uci.edu}
for more details.

{\tt Xpic, x2ps} and {\tt x2pic} are part of the user-contributed
software for the X Windows System Version 11 Release 3.  This document
is part of the source.

\end{document}             % End of document.
