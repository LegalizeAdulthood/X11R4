\chapter{Usage}
{\tt gwm [-mtDFia] [-x screens]  [-f profile] [-p path] [-d display]}

\section{Options}

The following command-line options\footnote{The options follow the
GETOPTS(3) conventions: they can be in any order, a space is optional between
an option and its argument; they can be combined (as in {\tt -Dmt}), but all
options must appear {\bf before} any argument, which is the managed display
for {\tt gwm}.} are supported:

\begin{description}

\item[-f filename] Names an alternate file as a {\GWM} startup file.
Default is {\bf .gwmrc.gwm} (note that the \verb".gwm" extension is
optional, as for any {\WOOL} file).\sloppy 

For instance, to use the {\sc Motif} emulation package, type
\verb"gwm -f mwm".

\item[-p~path] \label{GWMPATH}Gives the path to be searched for {\WOOL} or bitmap
files when loaded, including the startup file. Overrides the setting by the
environment variable {\bf GWMPATH}.  Defaults to \linebreak
\verb".:$HOME:$HOME/gwm:GWMDIR", where {\tt GWMDIR} is the local directory
where {\GWM} is installed (normally, {\tt /usr/local/lib/X11/gwm}).

You can append or prepend a path to the current path by preceding the path
given as argument to the \verb"-p" option by \verb"+" (for appending) or
\verb"-" (for prepending). For instance, if you want to search the directory
\verb"/usr/local/gwm" before the standard ones (including your homedir),
just say: \verb"-p -/usr/local/gwm".

\item[-d display] Specifies the name of the display whose windows must be
managed, such as \verb"unix:0". The \verb"-d" is optional, you can type
\verb"gwm unix:0".

\item[-x screens] Do not manage the screens given in the comma-separated
list of numbers, as in: {\tt -x 2,5,3}. Normally, {\GWM} manages {\bf all}
the screens of the given display.

\item[-1] Manages only the given screen, e.g. \verb"gwm -1 foo:0.2" manages
only the third screen of the display number \verb"0" on the \verb"foo"
machine. Same as defining the GWM\_MONOSCREEN shell variable.

\item[-D] Enables debugging mode for {\WOOL} files. In this version the only
effect of debug mode is to continue reading a file after a {\WOOL}
error occurred. Default behavior is to abort reading a file after an error.
Thus, if you modified your profile and introduced an error, {\GWM} will
refuse to complete execution, use {\tt gwm -D} to run it anyway.

\item[-s] Synchronize X calls, useful for debugging but slower.

\item[-t] Turns tracing on, as if you had done the call \verb"(trace t)" in
your profile.

\item[-q] (Quiet) Does not print startup banner, and sets the {\WOOL}
\verb"gwm-quiet" variable to 1.

\item[-m] Maps all toplevel windows already on screen. Useful after
unmapping some windows by accident.

\item[-F] Does not freeze the server during pop-up menus, move and resize of
the windows, which is the default behavior.

\item[-i] Disables the setting of input focus by {\GWM} (\verb"set-focus"
has no effects, except \verb"(set-focus ())", which resets the focus to
PointerRoot) on a window, keypresses go to the window under the cursor. Very
useful to debug profiles with only one screen.

\item[-a] Asynchronously handle moves and resizes, do not cancel a
move or a resize operation if the user released the button before the grid
appeared, which is the default behavior.

\item[-?] This, or any invalid option lists the available options and shows the
default path defined at compile time by your local installer.

\end{description}

\section{Shell variables}

{\GWM} can make use of the following shell variables:

\begin{description}

\item[GWMPATH] \sloppy for the path to search for files. If unset, defaults to
\verb".:$HOME:$HOME/gwm:GWMDIR", where {\tt GWMDIR} is the local directory
in which are installed all the standard {\GWM} files. (Normally {\tt
/usr/local/lib/X11/gwm}). Overridden by the \verb"-p" command line option.

\item[GWMPROFILE] for the name of the profile file. Defaults to
\verb".gwmrc.gwm". Overridden by the \verb"-f" command line option.

\item[DISPLAY] for the name of the X11 display to use, such as
\verb"unix:0.0". Overridden by the \verb"-d" command line option.

\item[GWM\_MONOSCREEN] if set will make {\GWM} manage only the given screen.

\end{description}

\section{Files}

{\GWM} needs at least one file for its startup, \verb".gwmrc.gwm" which must
be in {\GWM}'s path. New users do not need one, since a default profile should
already be present in the default path. The standard profile (see
\ref{standard-profile}, p~\pageref{standard-profile}) makes use of the
\verb".profile.gwm" file in the home directory.

The value of the default path can be printed by calling {\GWM} with the
\verb"-?" command line option.

The standard extensions used for {\GWM} file names are:
\begin{description}
\item[{\tt .gwm}] for {\WOOL} files.
\item[{\tt .xbm}] for {\bf X}11 {\bf B}it{\bf M}ap files, such as those
created with BITMAP(1).
\item[{\tt .xpm}] for {\bf X}11 {\bf P}ix{\bf M}ap files, which is an ASCII
portable format for distributing color images (see the \verb"pixmap-load"
function, p~\pageref{pixmap-load}).
\end{description}

\section{Description}

The {\GWM} command is a window manager client application of the X11 window
server specified in the {\tt display} argument (or the  {\tt DISPLAY} shell
variable if no argument is given), extensible via a built-in {\it lisp}
interpreter, {\WOOL} (Window Object Oriented Lisp) used to build {\bf Wobs}
(Window OBjects) which are used to decorate the windows of the other X11
applications running on the display. {\GWM}
tries to enforce the ICCCM conventions to communicate between X11 clients
and thus should be compatible with any well-behaved X11 application.

On startup, {\GWM} interprets its profile to build wobs describing how to
decorate user windows, which we will call {\bf Clients}.  Then it creates
{\bf Windows} around each client windows on the screens attached to the
managed display. A Window is made of 4 (optional) {\bf Bars} on the 4 sides
of the window.  Each of these bars is made of a variable number of {\bf
Plugs}, the most primitive wobs.  {\bf Menus} can then be made with a list
of bars. To each of these objects is associated a {\bf fsm} ({\bf F}inite
{\bf S}tate {\bf M}achine) describing their behavior in terms of {\WOOL}
code triggered by X or {\WOOL} events.

When {\GWM} wants to decorate a window, it calls the user-defined {\WOOL}
function \verb"describe-window" which must return a list of two window
descriptions (one for the window itself, and one for its icon) made by the
\verb"window-make" {\WOOL} primitive describing the window. To build these
descriptions the user can query the client window for any X11 properties and
use the X11 Resource Manager to decide how to decorate it.

The screens must also be described by such a description that {\GWM} will find
by calling the user-defined {\WOOL} function \verb"describe-screen" for each
managed screen.
