\chapter{The Standard GWM Packages}

\sloppy

This chapter describes the {\WOOL} packages available on the standard GWM
distribution. The name of the involved files are listed in the title of each
package.

{\GWM} does not try to enforce any policy in writing profiles, but for the
sake of simplicity and maintainability, all the {\WOOL} packages delivered
with {\GWM} will try to be compliant with a set of rules described in
section~\ref{standard-styleguide}, page~\pageref{standard-styleguide}, which
should assure the compatibility between them.

\section{The Standard Profile\hfill{\tt .profile.gwm}}
\label{standard-profile}

The standard profile can be customized to your taste by creating a
\verb".profile.gwm" file in your home directory, or by copying the one in
the {\GWM} distribution directory in your home directory and editing it.
\footnote{The standard profile is the default one, which you get if you do
not have a \verb".gwmrc.gwm" file in your home directory. It is a
real-estate overlapping environment. It is defined in the \verb".gwmrc.gwm"
file in the {\GWM} distribution directory.}

\subsection{mouse buttons}

The default behavior for clicking of the mouse buttons is, in a window
decoration or an icon:

\begin{description}

\item[left button] Moves the window. Releasing the button actually moves the
window, pressing another one while still holding down the left button
cancels the move operation

\item[middle button in a window] Resizes the window. The size of the window
will be displayed in the upper-left corner of the screen. Releasing the
button actually resizes the window, pressing another one while still holding
down the middle button cancels the resize operation

\item[middle button in an icon] de-iconifies the icon

\item[right button] brings up a pop-up menu for additional functions, such
as iconification and destruction

\end{description} 

These functions are enabled only in the {\GWM}-added decoration around the
window, or anywhere in the window if the {\bf alternate} (or {\bf meta} or
{\bf left} key on some keyboards) modifier key is depressed when clicking,
in the \verb"uwm" style of interaction.

Moreover, if you click in the icon in the upper left of the frame around the
xterm windows with the left or right button, the window will be iconified,
and with the middle button, it will be iconified and the icon moved just
underneath the pointer. You can still move the icon elsewhere by
dragging it while the middle button is down.

\subsection{customization}

Customization is achieved by creating a \verb".profile.gwm" file in
your home directory (or anywhere in your \verb"GWMPATH"). In this file
you can set the variables to modify the standard profile to suit your
taste. Note that you must set variables used in decorations {\bf
before} loading this decoration by a \verb"set-window" or
\verb"set-icon-window" call.

Your \verb".profile.gwm" will be loaded once for each screen
managed, and since pixmaps, colors, cursors and menus are screen-dependent
objects, try to define them as \verb"names" in the \verb"screen." namespace
(\seep{namespace-make}).

\subsubsection{Mouse bindings}

The standard mouse button bindings can be redefined by re-defining the
default {\bf states} (\seep{state-make}) for a
click in a window decoration, an icon and the root window, which are
respectively the global variables {\bf standard-behavior}, {\bf
icon-behavior}, {\bf root-behavior}. These states will be used to build the
{\bf fsms} of the windows, icons and root window. You may then need to
redefine the events grabbed by {\GWM}, in the {\bf grabs} variable.

\subsubsection{global switches}

The following global variables (which are names in the \verb"screen."
namespace for all pixmaps, colors, cursor and menus) controlling the way the
standard profile operates can be set in your \verb".profile.gwm" file:

\begin{description}

\itemtt{cursor} to the cursor displayed in any decoration or icon
\itemtt{root-cursor} to the cursor displayed on the root window. The
available cursors in the distribution are:
\begin{description} 
\itemtt{arrow} a big arrow
\itemtt{arrowhole} same with a hole inside
\itemtt{arrow3d} a 3d-looking triangular shape. Especially nice on a clear
background.
\end{description}
For instance, to use the ``{\tt arrow3d}'' cursor, just say:
{\exemplefont\begin{verbatim}
        (setq root-cursor (cursor-make "arrow3d"))
\end{verbatim}}

\itemtt{screen-tile} to the pixmap used to tile the root window with.
Provided bitmaps are \verb"back.xbm" (default) and \verb"grainy.xbm"

\itemtt{autoraise} if set to \verb"t", (defaults to \verb"()"), {\GWM} will
raise on top of others the window which has the input focus

\itemtt{xterm-list} the list of machines the user wants to launch a remote
xterm on (via the \verb"rxterm"\footnote{{\tt rxterm}, {\tt rxload}
and {\tt rx} are Bourne shell scripts used to start remote xterms,
xloads or any other X command. The rxterm script is included in the
distribution, in the {\tt gwm} subdirectory, install it and make rx
and rxload as links to it} command)

\itemtt{xload-list} the list of machines the user wants to launch a remote
xload on (via the \verb"rxload" command) 

\itemtt{icon-pixmap} the pixmap to be displayed in the upper left
corner of window to iconify it 

\itemtt{to-be-done-after-setup} the list (\verb"progn"-prefixed) of things
to be executed after all windows already present have been decorated.  

\itemtt{look-3d} to \verb"t" to specify that window decoration packages that
support it should adopt a tridimensional look. The default for this variable
is \verb"()" on monochrome displays and \verb"t" on color and grayscale ones.

\end{description}

\subsubsection{Window and icon decoration}

Moreover, you can decide to change the decoration (look and feel)
of a client or an associated icon by using the following functions.

{\bf Note:} for all functions, to set defaults for a screen type or a client
class, use the \verb"any" keyword.

\begin{description}

\itemtt{(set-window [screen-type] client-class decoration)}
\label{set-window} will tell {\GWM} to use the decoration described in {\tt
decoration} for all clients of client class {\tt client-class} in the
current screen of type \verb"screen-type". {\tt decoration} can be either:

\begin{itemize} 
\item a real decoration made with a {\tt window-make} call
\item a function returning a decoration when called without parameters 
\item the file name (as a string or atom) of one of the standard decorations 
as listed in the section \ref{standard-decorations}. The file will be loaded
and should set the {\tt decoration} atom to either a real decoration or a
function returning a decoration 
\end{itemize}

For instance, these declarations say that xterms on this screen, if it is a
monochrome one will be decorated by the \verb"simple-ed-win" style, but if
the screen is a color or grayscale one, will use the \verb"simple-win"
decoration, and will use the ``no-decoration'' style for windows not otherwise
describe (The {\tt no-decoration} window description adds no visible
decoration to a window).

{\exemplefont\begin{verbatim}
        (set-window mono XTerm simple-ed-win)
        (set-window any XTerm simple-win)
        (set-window any no-decoration)
\end{verbatim}}

To choose decorations on other criteria than just class, define a function
that will return a {\WOOL} expression which will give the good decoration when
evaluated. For instance, to put a \verb"simple-ed-win" decoration on all
XTerms, except those less than 200 pixels wide, use this in your .profile.gwm:

{\exemplefont\begin{verbatim}
        (defun decide-which-xterm-deco ()
            '(if (< window-width 200) (no-frame)
                 (simple-ed-win)))
        (set-window any XTerm decide-which-xterm-deco)
\end{verbatim}}

\itemtt{(set-icon [screen-type] client-class bitmap-file)} \label{set-icon}
will associate to a client a simple icon made of the the X11 bitmap stored
in \verb"bitmap-file" (with the current value of \verb"foreground" and
\verb"background" at the time of the call to \verb"set-icon") and the name
of the icon underneath it. 

If a list is given as the last argument, it is evaluated and the pixmap is
taken as the result of the evaluation.

Suppose that you designed a picture of a mailbox named ``mail-icon.xbm'',
saved it somewhere in your GWMPATH, and want to use it for the icon of the
client \verb"xmh". You would add in your \verb".profile.gwm" one of the two
forms:

{\exemplefont\begin{verbatim}
        (set-icon XMh mail-icon.xbm)
        (set-icon XMh (pixmap-make black "mail-icon.xbm" white))
\end{verbatim}}

{\bf Note:} the icon bitmap can only be set for icon decorations
supporting it, such as the (default) \verb"simple-icon" decoration style.

\itemtt{(set-icon-window client-class icon-file)}\label{set-icon-window} This
call will associate more complex icons to a given client, such as those
listed in section~\ref{standard-icons}, p~\pageref{standard-icons}. For
instance to have \verb"xterm" icons look like a mini computer display, add
in your \verb".profile.gwm":

{\exemplefont\begin{verbatim}
        (set-icon-window XTerm term-icon)
\end{verbatim}}

The \verb"term-icon" argument can be either a client window decoration, a
function returning a decoration or a file name, as for the \see{set-window}
function. On startup, {\GWM} does a:

{\exemplefont\begin{verbatim}
        (set-icon-window any any simple-icon)
\end{verbatim}}

\end{description}

\subsubsection{Desktop space management\hfil{\tt placements.gwm}}

The standard profile supports functions to automatically place your windows
or icons on the screen. These functions manage only some type of clients (or
all of them if affected to the \verb"any" client), and they are called
with one argument set to {\bf t} on opening the window, and to {\bf ()} on
closing (destruction) it. They are associated to clients by the calls:

\begin{description}

\itemtt{(set-placement [screen-type] client-class function-name)} for the
main windows of a client. 

\itemtt{(set-icon-placement [screen-type] client-class function-name)} for
its associated icons.

\end{description}

The currently pre-defined placement functions are:

\begin{description}

\itemtt{()} does nothing, the window justs maps where it was created
by the client (this is the default value)

\itemtt{user-positioning} asks the user to place it interactively

\itemtt{rows.XXX.placement} automatically aligns the windows or icons on the
sides of the screen. Replace {\tt XXX} by one of the eight names in the 
following figure:

\centerline{\texpsfig{places1.id}{261}{117}}

You can set the space where the {\tt XXX} row lives by issuing calls to the
control function \verb"rows.limits" whith the syntax:

\begin{verbatim}
        (rows.limits rows.XXX [key value] ...)
\end{verbatim}

where {\tt key} is an atom setting a value (in pixels). Key can be either
start, end, offset, and separator as shown in the following figure for 
{\tt XXX = top-left}:

\centerline{\texpsfig{places2.id}{195}{90}}

\end{description}

For instance you can manage your \verb"xterm" windows by:

{\exemplefont\begin{verbatim}
        (set-placement XTerm user-positioning)
        (set-icon-placement any XTerm rows.right-top.placement)
        (rows.limits rows.right-top 'start 100 'separator 2)
\end{verbatim}}

You can define your own window placement functions and use them by the
\verb"set-placement" call. They will be called with one argument, {\bf t}
when the window (or icon) is first created, and {\bf ()} when the window is
destroyed. This is why we needed an interpretive extension langage!

{\exemplefont\begin{verbatim}
        (defun do-what-I-mean (flag) ...great code...)
        (set-icon-placement any any do-what-I-mean)
\end{verbatim}}

\subsubsection{Menus}

The displayed menus can be redefined by setting the following global
variables to menus made with the choosen menu package. The default package
is the \seeref{std-popups} package.

\begin{description}
\itemtt{window-pop} for the menu triggered in windows
\itemtt{icon-pop} for the menu triggered in icons
\itemtt{root-pop} for the menu triggered in the root window
\end{description}

You can look at their standard implementation in the \verb"def-menus.gwm"
distribution file.

\section{Duane Voth's rooms\hfill{\tt dvrooms.gwm}}
\label{dvroom}

{\bf Duane Voth} (\verb"duanev@mcc.com") made a mini {\em rooms} 
package to manage groups of windows. To use it, put in your .profile.gwm
the line
\begin{verbatim}
        (load "dvrooms")
\end{verbatim}
before any calls to any \verb"(set-..." call.
Then, with the standard profile, you can add new rooms by the root pop-up menu,
or by explicitly calling the \verb"new-dvroom-manager" function in your
profile, with the name of the room as arguments.
{\exemplefont\begin{verbatim}
        (new-dvroom-manager "mail")
        (new-dvroom-manager "dbx")
\end{verbatim}}
The name of the room itself is the same editable plug as the one used for
the \verb"simple-ed-win" window decoration, so that you can edit it
by double-clicking or control-alt clicking with the left button.

Only one room is ``open'' (non-iconified) at a time, and calling the
functions \verb"add-to-dvroom" or \verb"remove-from-dvroom" (from the
window menu or from {\WOOL}) on a window will add or remove it from the group
of windows that will be iconified or de-iconified along with the room manager.
Opening a room will close the previously open one, iconifying all its managed
windows. New rooms start as icons.

This package will recognize as a room manager any window with the name 
\verb"rmgr". You can then create new rooms by other Unix processes.

Context used:
\begin{description}
\itemtt{dvroom.font} font of the name of the room
\itemtt{dvroom.background} background color
\itemtt{dvroom.foreground} color of the text of the name
\itemtt{dvroom.borderwidth} borderwidth of the room
\itemtt{dvroom.x, dvroom.y} initial position of the room
\itemtt{dvroom.name} string used to build the name of the room. Defaults
to \verb|"Room #"| (a number will be concatenated to it).
\itemtt{edit-keys.return, edit-keys.backspace, edit-keys.delete} keys used
for editing, see \verb"simple-ed-win", p~\pageref{simple-ed-win}
\end{description}

\section{Group Iconification\hfill{\tt icon-groups.gwm}}
\label{icon-groups}

Loading the {\tt icon-groups} package redefines the {\tt iconify-window}
function to use only one icon for all windows of the same group. Iconifying
the group leader will iconify all the windows in the group, whereas
iconifying a non-leader member of the group will only unmap it and map the
common icon if it is not already present.

\section{Standard pop-up menus\hfill{\tt std-popups.gwm}}
\label{std-popups}

This package implements a very simple pop-up menu package. A menu is created
by the \seeref{menu-make} call, with the help of two functions to create the
items of the menu:

\begin{description}

\itemtt{pop-label-make} to create an inactive label on top of the menu, by
\verb|(pop-label-make Label)| where \verb"Label" is the string to be
displayed as the title of the menu.

\itemtt{item-make} to create a label triggering a {\WOOL} function call, by
\verb|(item-make Label Expr)| where \verb"Label" is the string to be
displayed as the item of the menu, and \verb"Expr" will be evaluated when
releasing the button in the item.

\end{description}

For instance, the default root menu is made by the call:

{\exemplefont\begin{verbatim}
        (setq root-pop
            (menu-make
                (pop-label-make "Gwm")
                (item-make "Xterms..." (pop-menu xterm-pop))
                (item-make "refresh" (refresh))
                (item-make "restart" (restart))
                (item-make "reload" (load ".gwmrc"))
                (item-make "Xloads..." (pop-menu xload-pop))
                (item-make "Wool infos" 
                    (progn (hashinfo)(gcinfo)(wlcfinfo)(meminfo)))
                (item-make "Exec cut" 
                    (execute-string (+ "(? " cut-buffer ")")))
                (item-make "End" (end))
            ))
\end{verbatim}}

Moreover, you can control the appearance of the label and the items of the menu by the following variables:

\begin{description}
\itemtt{pop-item.font} font of the items
\itemtt{pop-item.foreground} color of their text
\itemtt{pop-item.background} color of their background
\itemtt{pop-label.font} font of the labels on top of menus
\itemtt{pop-label.foreground} color of their text
\itemtt{pop-label.background} color of their background
\end{description}

\section{Gosling Emacs mouse support\hfill{\tt emacs-mouse.gwm}}
\label{emacs-mouse}

This package implements a way to use the mouse with the {\bf Gosling} emacs,
sold by {\bf Interpress}. You will need to load the emacs macros contained
in the {\tt gwm.ml} file included in the distribution in your emacs. Then,
in a window decorated with the {\tt simple-ed-win} package, pressing
{\bf Control} and:

\begin{description}
\item[left mouse button] will set the emacs text cursor under the mouse
pointer
\item[middle mouse button] will set the mark under the mouse pointer
\item[right mouse button] will pop a menu of commonly-used emacs functions
(execute macro, cut, copy, paste, go to a C function definition, re-do last
search)
\end{description}

Clicking in the mode lines will do a full screen recursive edit on the
buffer if not in a target and in the targets:

\begin{description}
\item[{\tt [EXIT]}] will delete the window if it is not the only one on
the screen, or do an {\tt exit-emacs}
\item[{\tt [DOWN]}] will scroll one page down the file
\item[{\tt [ UP ]}] will scroll one page up the file
\end{description}

This package is included rather as an example of things that can be done
to work with old non-windowed applications than as a recommended way of
developping code.

\section{Sample window decorations\hfill{\tt *-win.gwm}}
\label{standard-decorations}

These are standard window decorations which can be used via the
\verb"set-window" function of the standard profile. They can be found in
files whose names end in {\tt -win.gwm} in the distribution directory of
{\GWM}.

\subsection{Simple window\hfill{\tt simple-win.gwm}}
\label{simple-win}

\centerline{\texpsfig{simple-win.ps}{168}{75}}

This is a really simple window decoration with only a title bar on top of
the window. The name of the window is centered in the bar and the bar
changes its background color to show which currently has the input focus.

This style is customizable by setting the following variables at the top
of your .profile, before any call to \verb"set-window":

\begin{description}
\itemtt{simple-win.font} to the font used for printing the title.
\itemtt{simple-win.active} to the color of the top bar for the window having
the keyboard focus (defaults to darkgrey).
\itemtt{simple-win.inactive} to the color of the top bar when the window
do not have the keyboard focus (defaults to grey).
\itemtt{simple-win.label.background} background color of the name.
\itemtt{simple-win.label.foreground} color of text of the name.
\end{description}

\subsection{Simple editable window\hfill{\tt simple-ed-win.gwm}}
\label{simple-ed-win}

\centerline{\texpsfig{simple-ed-win.ps}{176}{89}}

This decoration has a titlebar on top of it, including an iconification
plug and an editable name plug. Moreover the whole border changes color to
track input focus changes.The look of this
frame can be altered by setting the following variables:

\begin{description}
\itemtt{icon-pixmap} to a pixmap used as an iconification button.
\itemtt{simple-ed-win.borderwidth} to the width in pixels of the 
sensitive border of the window
\itemtt{simple-ed-win.font} to the font used for printing the title.
\itemtt{simple-ed-win.active} to the color of the top bar and border
 for the window having the keyboard focus (defaults to darkgrey).
\itemtt{simple-ed-win.inactive} to the color of the top bar and border
when the window do not have the keyboard focus (defaults to grey).
\itemtt{simple-ed-win.label.background} background color of the name.
\itemtt{simple-ed-win.label.foreground} color of text of the name.
\end{description}

When you click in the icon button at the left of the titlebar (whose pixmap
can be redefined by setting the global variable \verb"icon-pixmap" to a
pixmap) with the left button, the window is iconified. If you click with the
middle button, you will be able to drag the ouline of the icon and release
it where you want it to be placed.

If you double-click a mouse button, or do a click with the {\bf control} and
{\bf alternate} keys depressed, in the editable name plug at the right of
the titlebar, you will be able to edit the name of the window (and the
associated icon name) by a simple keyboard-driven text editor whose keys
are are given as strings in the following variables:

\begin{description}
\itemtt{edit-keys.return} to end edition (defaults to \verb|"Return"|).
\itemtt{edit-keys.delete} to wipe off all the text (defaults to 
\verb|"Delete"|).
\itemtt{edit-keys.backspace} to erase last character (defaults to 
\verb|"Backspace"|).
\itemtt{{\em any key}} to be appended to the text.
\end{description}

The plug will invert itself during the time where it is editable. You quit
editing by pressing {\bf Return}, double-clicking in the plug or exiting the
window.

This decoration style also supports the \verb"emacs-mouse" package.

\subsection{Frame\hfill{\tt frame-win.gwm}}
\label{frame-win}

\centerline{\hfil
\texpsfig{frame-3d.ps}{85}{85}\hfil
\texpsfig{frame-2d.ps}{72}{72}\hfil}

This decoration consists in a frame around the window. The look of this
frame can be altered by setting the following variables:

\begin{description}
\itemtt{look-3d} to {\tt t} to have a ``3D-looking'' frame (left figure) 
instead of the 2d-looking one (right figure).
\itemtt{frame.top-text} to an object which will be evaluated to yield the
text to be put on top of the frame
\itemtt{frame.bottom-text} to an object which will be evaluated to yield the
text to be put on the bottom of the frame, for instance:
{\exemplefont\begin{verbatim}
        (setq frame.bottom-text '(machine-name))
\end{verbatim}}
\itemtt{frame.pixmap-file} to the prefix of the 8 bitmaps (or pixmaps) files
used to build the frame. The suffixes will be {\tt .tl .t .tr .r .br .b .bl
.l}, clockwise from upper left corner.
\itemtt{frame.pixmap-format} to the format of the files: \verb"'bitmap"
(default) or \verb"'pixmap"
\itemtt{frame.bar-width} to the width of the four bars (should match the
pixmap files)
\itemtt{frame.inner-border-width} to the inner border width
\end{description}

\section{Sample icons\hfill{\tt *-icon.gwm}}
\label{standard-icons}

These are standard icon windows descriptions which can be used via the
\verb"set-icon-window" function of the standard profile. They can be found in
files whose names end in {\tt -icon.gwm} in the distribution directory of
{\GWM}.

\subsection{Simple icon\hfill{\tt simple-icon.gwm}}
\label{simple-icon}

This icon consists in an (optional) image and the icon-name of the
window below it. The image is by priority order:

\begin{description}
\item[the user pixmap] set by the \seeref{set-icon} call
\item[the window] the client has set in its hints to the window manager to
use as an icon
\item[the pixmap] the client has set in its hints to the window manager to
use for its icon
\end{description}

Used context variables:

\begin{description}
\itemtt{simple-icon.font} for the font used to display the icon-name.
\end{description}

\subsection{Terminal display icon\hfill{\tt term-icon.gwm}}
\label{term-icon}

\centerline{\hfil
\texpsfig{term-icon1.ps}{42}{34}\hfil
\texpsfig{term-icon2.ps}{66}{34}\hfil}

This icon looks like a small computer display with the window name inside
it. Note that the icon resizes itself to adjust to the dimensions of the
displayed name.

Used context variables:

\begin{description}
\itemtt{term-icon.font} for the font used to display the icon-name.
\end{description}

\section{Utilities\hfill{\tt utils.gwm}}
\label{utils}

This package implements some useful functions for the {\WOOL} programmer.
List this file to see the current ones.

\section{Programming Styleguide for the standard distribution}
\label{standard-styleguide}

The styleguide to write decorations is to be written. Until then, look at
existing files such as {\tt .gwmrc.gwm, .profile.gwm, simple-ed-win.gwm} to
see what is the current style. We will appreciate all feedback to these
conventions, which are not settled yet.

The main idea is that a decoration package \verb"foo" should, when loaded,
define a \verb"foo" function which, once executed will return the appropriate
decoration, using the pre-defined behaviors.

All persistent variables of the packages should be prefixed by the package
name, as in \verb"foo.bar".

Do not forget to define in fact one decoration per screen or be cautious not
to mix colors, pop-ups, pixmaps and cursors from screen to screen. Use 
\verb"defname screen." to declare screen-specific variables.

The user should be allowed to customize the decoration by setting global
variables in its \verb".profile.gwm" file, which will be interpreted during
the loading of your package.

The fsm you make for your package should be contructed from the behaviors
defined in \verb".gwmrc.gwm", such as standard-behavior,
standard-title-behavior, window-behavior, icon-behavior, root-behavior, or
already built fsms such as fsm, window-fsm, icon-fsm, root-fsm.

All values you want to attach to windows or wobs should be put as properties
in the property-list of the wobs, by calls to 
\verb"(## 'key wob value)"

The main idea is, if you must modify {\tt .gwmrc.gwm} to code your window
decoration, mail us your desiderata and/or enhancements, so that we should
be able to keep the same .gwmrc for all decorations.
I maintain mailing lists for people to exchange ideas about {\GWM}. Mail me
a request if you are interested at \verb"gwm@mirsa.inria.fr".

\section{Other profiles}

Other nice profiles have also been developed in parallel to the standard
profile, but they have not been integrated yet, i.e. they need their own
{\tt .gwmrc.gwm}.

\subsection{The MWM emulation package\hfill{\tt mwm.gwm}}

{\bf Frederic Charton} made this profile emulating the {\bf M}otif
{\bf W}indow {\bf M}anager. To use it, you must give the command line 
option {\bf -f mwm} to {\GWM}.

You can customize it by copying in your
\verb"gwm" directory (in you GWMPATH) the files:

\begin{description}
\itemtt{mwmrc.gwm} all the resources settable in ``.mwmrc'' for {\sc Mwm}
are also settable here, except for the menus.
\itemtt{mwm-menusrc.gwm} description of the menus.
\itemtt{mwmprofile.gwm} miscellaneous wool customizations (needs {\WOOL}
knowledge)
\end{description}

and editing it. You may want to get the {\sc Mwm} manual for the description
of all the available functions. For instance, to set the input focus management
from ``click to type'' (default) to ``real estate'' (input focus is always
to the window underneath the pointer), edit \verb"mwmrc.gwm" and change
the line \verb"(: keyboardFocusPolicy 'explicit)" to 
\verb"(: keyboardFocusPolicy 'pointer)".

{\bf Warning:} This profile is still mono-screen, i.e. to manage 2 screens
on your machine, you must run 2 gwms, for instance by:
{\exemplefont\begin{verbatim}
        gwm -1 -f mwm unix:0.0 &
        gwm -1 -f mwm unix:0.1 &
\end{verbatim}}

\subsection{The TWM emulation package\hfill{\tt twm.gwm}}

{\bf Arup Mukherjee} (\verb"arup@grasp.cis.upenn.edu") made a {\sc
Twm} emulator. To use it, you must give the command line 
option {\bf -f twm} to {\GWM}.

You can customize it by copying in your \verb"gwm"
directory (in you GWMPATH) and editing the files:

\begin{description}

\itemtt{twmrc.gwm} Contains numerous options (mainly colors) that can be
set from here. The file is well commented, and most of the color
variables have self-explanatory names. You can also specify from here
whether or not the icon manager code is to be loaded.
It also contains definitions for the three variables
\verb"emacs-list", \verb"xterm-list", and \verb"xload-list". 
The specified hostnames are
used to build menus from which you can have gwm execute the respective
command on a host via the ``rsh'' mechanism (note that your .rhosts
files must be set up correctly for this to work). Note that unlike
with the standard profile, the rxterm and rxload scripts are {\bf NOT}
used.

\itemtt{twm-menus.gwm} The contents of all the menus are specified
here. To change more than the xterm, xload, or emacs lists, you should
modify this file.

\itemtt{twm.gwm} The only things that one might wish to customize here are the
behaviors (which specify the action of a given button on a given
portion of the screen)
\end{description}

\section{Troubleshooting}

To debug a {\WOOL} program, you can:

\begin{itemize}

\item use the \verb"trace" function to trace code execution or evaluate an
expression at each expression evaluation.

\item read, execute, and print {\WOOL} code by selecting it and using
the {\bf Exec cut} (for execute cut buffer) menu function.

\item use the \verb"-s" command line option to synchronise X calls, if you want
to know where you issue a non-legal X call.

\item compile {\GWM} with the -DDEBUG compile option, which will turn on many
checks (stack overflow, malloc checks, etc...) in case you manage 
to make {\GWM} crash.

\end{itemize}
