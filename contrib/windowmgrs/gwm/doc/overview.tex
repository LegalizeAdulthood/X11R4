\chapter{Overview}

\sloppy

{\GWM} like any other window manager is in charge of all that is {\bf
exterior} to other application windows on the display. It performs its task
by {\bf decorating} the existing windows on the screen with its own windows.
The appearance and behavior of these windows are described by the user through
programs written in the lisp-like {\WOOL} language, which are interpreted by
the built-in {\WOOL} interpreter of {\GWM}.

The {\GWM} Window manager is composed of 2 modules: the window manager itself,
{\GWM}, and the lisp interpreter {\WOOL}. The role of the garbage-collected
{\WOOL} objects is to build shared descriptors that will be used by {\GWM}
to build its (non-shared and non-garbage-collected) objects.

\section{GWM Window OBjects: the wobs}

{\GWM} is built upon the notion of a {\bf wob}, which, not unlike an X {\sl
Widget} is an object composed of:

\begin{description} 

\item[an X11 window] used to display the wob on the screen (output)

\item[a {\WOOL} finite state machine] used to trigger {\bf WOOL}
functions in response to X11 events sent to the wob (input)

\end{description}

Wobs are not directly created with {\WOOL} contructors, they are {\bf
described}, and {\GWM} uses this description to physically create the
wobs when it needs to, when decorating a new client window for instance.

Like any X11 window, the user can choose the width, color(s), and tiling
pixmap of the border of the wob (which is considered to be in the wob for
input purposes). There are four kinds of wobs:

\begin{description}

\item[Window] An X11 application, such as XTERM(1) or XCLOCK(1) usually opens
one or more windows directly on the screen (in fact the {\em root window}).
We will call these windows, which are not created by {\GWM}, {\bf client
windows}.  {\GWM} will ``decorate'' these client windows by {\em
reparenting} them, i.e. by creating a {\bf Window} wob and making the client
window a son of this newly created window.

A window is a wob made by creating a new toplevel X window, reparenting  the
client window as its child, and framing it with four (optional) {\em bars},
children of the new toplevel window. Note that the inside of a window wob is
thus never visible, since it is entirely covered by the bars and the client
window.

\item[Bar] The only extensible wob, it has a width and an extensible length.
It is a line (vertical or horizontal) of {\em plugs} centered on the axis of
the bar with optional stretchable space between them.

\item[Plug] The simplest of all wobs, its contents is just a graphic
which is displayed in their X window. It acts thus like some kind of button.
Current graphics are text and pixmaps.

\item[Menu] A bar of bars (horizontal menus consist of a
horizontal bar of vertical bars, vertical menus are a vertical bar of
horizontal ones).

Menus are the only ``stand-alone'' wobs, their windows are directly created
by {\GWM} on the screen. They can be used to implement pop-up, pull-down or
fixed menus, control panels, message windows and icon boxes.

\end{description}

\subsubsection{Other GWM objects}

Other gwm objects are just X objects (fonts, pixel colors ...) that are
referenced by their X id, and are accessed via encapsulating
{\WOOL} types, such as Numbers or Pixmaps.

\section{WOOL objects}

{\WOOL} is a lisp interpreter of a special kind, since it has some important
design differences from ``real'' lisps:

\begin{description}

\item[incremental garbage collection] {\WOOL} has a {\sl reference-count}
driven memory management scheme. This means that the load of memory
management is evenly distributed on all phases of computing, thus avoiding
the dreadful garbage collection pauses. But, since reference count memory
management doesn't work with circular lists, no {\WOOL} function allows the
user to do physical replacements on lists\footnote{In fact, some do, but are
flagged as ``for experts only'' in the documentation}. 

\item[no real lists, but arrays] {\WOOL} lists are internally stored as
arrays, speeding up list scanning. We do not really need the
generality of the chained cells model, since we do not want to have circular
lists.

\item[monovalued atoms] In classical lisp dialects, you can give a variable
and a function the same name without conflict. In {\WOOL}, an atom can only
have one value.

\item[internally typed objects] All of the {\WOOL} interpreter is written in
an object oriented way. This is not visible to the user, since we do not
offer a way to define user types, but it greatly improves the modularity of
the code, and allows us to provide generic functions, such as the \verb"+"
function operating on numbers, lists, or strings.

\end{description}

The different {\WOOL} objects are:

\begin{description}

\item[atoms] associates any {\WOOL} object to any other via a hash table.
Only one wool object can be referenced, which implies that a \verb"(setq foo
1)" assignment will remove any function definition made by \verb"(defun foo
...)". There is {\bf no} limit on the length of atom names. Atom names can
be composed of any printable ASCII character except {\tt " ' ( )}, and
cannot begin by a digit.

\item[active values] are predefined atoms that can be used as atoms or
functions. If \verb"foo" is an active value, then \verb"foo" and
\verb"(foo)" give the same result, and the calls \verb"(setq foo obj)" and
\verb"(foo obj)" have the same effect.

Active values are just like other atoms, but setting and evaluating them
trigger specific functions to allow for greater expressive power. For
instance, just setting a wob's \verb"borderwidth" will actually resize the
corresponding X window's border, and declaring a local value for it by a
\verb"(with (wob-borderwidth 3) ...)" will actually change the borderwidth
of the current wob on the screen during the execution of the \verb"with"
body, and then revert to the previous value.

\item[numeric variables] are atoms that can only store integer values. Like
active values, they can be used as variables or functions. Setting them to 
\verb"()" or \verb"t" equivalent to setting them to \verb"0" or \verb"1".

\item[namespaces] are sets of variable names that have a different value for 
each state of the namespace. For instance, the most useful namespace is the
\verb"screen." one, having one state for each screen. So each name in this
namespace, such as \verb"white", can hold a screen-dependent value that will
always eval to the correct value relative to the screen.

\item[integers] are 32-bit signed integer values.

\item[strings] are 8-bit strings of characters, with no size limit.

\item[functions] may evaluate or not their arguments and have a fixed
or variable arity.

\item[fsms] the finite state machines. They are {\WOOL} objects shared by
wobs and respond to both X events and {\WOOL}-made events, the so-called
``user'' events.

\item[wobs descriptors: plug, bar, menu, client] are used by {\GWM} to build
its wobs.

\item[X objects: pixmaps, cursors, events, labels] allow X
resources to be shared via the {\WOOL} memory management.

\item[internal objects] used to improve efficiency (These include
collections and quoted expressions).

\end{description}

\section{Operation}

When you start {\GWM}, it:

\begin{itemize}

\item connects to the display it must manage to initialize its X11 structures

\item reads and interprets the user's {\WOOL} profile (searched for in a
user-defined path, see usage). This profile must define at least two
{\WOOL} functions:

\begin{description}
\item[describe-window] which will be called by {\GWM} to know how to decorate
any client window and must return a list of two window descriptors, one for
the window itself, and one for the associated icon;
\item[describe-screen] which will be called by {\GWM} for each managed
screen and must return a window descriptor;
\end{description}

\item looks to see if it is the only window manager running; if not, it
aborts.

\item decorates the managed screens by calling the user-defined {\tt
describe-screen} function for each one, with the \verb"screen" active value
being set to the current screen.

\item decorates all already mapped client windows, by calling {\tt
describe-window} with the current client window being set to each window.

\item executes the (user-defined) opening function for each screen.

\item enters {\GWM} main loop, which consists in:

\begin{itemize}
\item waiting for an X event

\item examining the received event, and if it is for an existing wob, sends
it to the {\em fsm} of this wob, else if it is a new window which is being
mapped for the first time, decorates it (by calling {\tt describe-window})
\end{itemize}

When an event is sent to a {\em fsm}, it is matched against the
transitions in the current state of the fsm, and as soon as one matches, the
corresponding {\WOOL} expression is evaluated, and the fsm changes state if
necessary. If no transition is waiting for the event, it is discarded.

\end{itemize}

\section{Lazy evaluation}

For sub-wobs of wobs, i.e., bars of a window, plugs of a bar, bars of a
menu, and menu of any wob, lazy evaluation is performed. That is, on the
realization of the wob the field is re-evaluated if the result is not
of the expected type. This allows for constructs such as:

{\exemplefont\begin{verbatim}
        (plug-make '(label-make window-name))
\end{verbatim}}

which creates a plug whose text is fetched as the name of the
window on each realization, you do not have to explicitly \verb"eval" a
quoted expression.

\section{Screen-dependent objects}

An invocation of {\GWM} can manage all screens attached to a display (there
is one keyboard and mouse per display), but in X, screens are very distinct
worlds. If you create a pixmap or a color on a screen, you cannot use it on
another one. The list of objects created on one screen that cannot be used
on another is:

\desc{pixmaps}{Objects}{made by}{
colors & color-make\\
pixmaps & pixmap-make, label-make, active-label-make\\
cursor & cursor-make\\
menus & menu-make\\
}

And of course, all the wobs and windows are screen-specific.

