%exemple begin at col 9, ==> at col 41
\chapter{WOOL Reference manual}
\sloppy

\ITEMa{;}{{\WOOL} comment}

{\usagefont\begin{verbatim}
; any text up to the end of line
\end{verbatim}}\usageupspace

Comments in {\WOOL} begin with a semicolon and end at the end
of the line.

\ITEMa{!}{executes a shell command}

{\usagefont\begin{verbatim} 
(! command arguments...)
\end{verbatim}}\usageupspace 

Executes (forks) the  command given in string with its given arguments, and
does not wait for termination (So you don't need to add a  final ``\&''). For
instance, to open a new ``xterm'' window on the machine where {\GWM} is
running, execute \verb|(! "xterm")|. This is an interface to the
\verb"execvp" call: the command is searched for in the \verb"PATH" variable,
and executed via \verb"/bin/sh" if it is a command file, directly executed
otherwise.

{Examples:\exemplefont\upspace\begin{verbatim}
        (! "xterm")                       ; machine-executable code
        (! "rxterm" "foobar")             ; or shell script
        (! "xterm" "-display" "bar:0.1")  ; with arguments
        (! "/bin/sh" "-c" "for i in a b c;do xclock -display $i:0;done")
                                          ; needs a shell for commands
\end{verbatim}}

\ITEMbi{\#}{nth}{accesses an element of a list}

{\usagefont\begin{verbatim}
(# n list [object])
(# atom list [object])
\end{verbatim}}\usageupspace

Gets (or sets to \verb"object" if present) the n-th element of the list
\verb"list" (starting with 0). Increases the list size if needed (with
nils).  When    the first argument is an atom, this references the element
just after the atom in the list, thus providing the ability to maintain
classical Lisp ``property lists''.

Note that this function does not do a physical replacement, and constructs a
new list with the \verb"list" argument. \seesnp{replace-nth}

{Examples:\exemplefont\upspace\begin{verbatim}
        (# 2 '(a b c d))                ==>     c
        (# 2 '(a b c d) 'foo)           ==>     (a b foo d)
        (# 'x '(x 4 y 6))               ==>     4
        (# 'y '(x 4 y 6) 8)             ==>     (x 4 y 8)
        (# 'z '(x 4 y 6) 10)            ==>     (x 4 y 6 z 10)
        (# 6 '(a b c d) 'foo)           ==>     (a b c d () () foo)
\end{verbatim}}

In fact the index can be any Lisp object, but as the list is scanned to find
the {\bf same} object (EQ predicate), using atoms is the safest way.

{\exemplefont\begin{verbatim} 
        (# "foo" '("foo" 1))            ==>     ()
        (progn (setq name "foo")
                (# name (list name 1))) ==>     1
\end{verbatim}}

{\bf Note:} The second argument can also be a:
\begin{description}
\item[wob] in which case the contents of the \verb"wob-property" is taken
as the list.
\item[symbol] which must evaluate to a list, which is then used by the
function.
\end{description}

\ITEMbi{\#\#}{replace-nth}{physically replaces an element of a list\hfill{\bf EXPERT}}

{\usagefont\begin{verbatim}
(## n list object)
(## atom list object)
\end{verbatim}}\usageupspace

This function physically replaces an element of the list, which is located
as with the \verb"#" function. It returns the modified list. 

This provides a way to implement variable-sized lists, such as linked lists,
which are the real lists in the Lisp world, but should be used with care, as
you are able to create circular references with it. To allow the {\WOOL}
garbage collector to handle correctly circular lists, you should always
explicitly set to () the fields of a circular cell before disposing of it.

Example: a circular list with cells of the form \verb"(car cdr)"
{\exemplefont\begin{verbatim}
        (setq elt2 '(b ()))             ; second cell of list
        (setq elt1 (list a elt2))       ; first one pointing to second
        (## 1 elt2 elt1)                ; we make a circular list
\end{verbatim}}
Now, if we set elt1 and elt2 to \verb"()", their storage will not
be freed!, we must do
{\exemplefont\begin{verbatim}
        (## 1 elt2 ())
\end{verbatim}}
before setting them to \verb"()".

{\bf Note:} The second argument can also be a:
\begin{description}
\item[wob] in which case the contents of the \verb"wob-property" is taken
as the list.
\item[symbol] which must evaluate to a list, which is then used by the
function.
\end{description}
This is the {\bf only} case where lists can be extended, by specifiyng 
an atom which do not exist in the list. In this case, if the list is pointed
to by many objects, the list is duplicated, and only the copy pointed to by
the wob or atom argument is modified, e.g.:

{\exemplefont\begin{verbatim}
        (setq l '(a 1))
        (setq l-bis l)                  ; l and l-bis points to the same list
        (## 'b 'l 2)                    ; only l is modified in place
        l                               ==> (a 1 b 2)
        l-bis                           ==> (a 1)
\end{verbatim}}

\ITEMa{(}{list notation}

{\usagefont\begin{verbatim}
(elt1 elt2 ... eltN)
\end{verbatim}}\usageupspace

This notation is used to describe a {\bf list} of {\tt N} elements, as in
other Lisp languages. Note that {\WOOL} is not a true Lisp dialect since lists
are represented internally as arrays, allowing for a greater speed and a
smaller memory usage in the kind of programs used in a window manager. If
you are a die-hard Lisp addict, you can still use CONS, CAR and CDR
if you want by defining them as:

{\exemplefont\begin{verbatim}
        (defun cons (e l) (+ (list e) l))
        (defun car (l) (# 0 l))
        (defun cdr (l) (sublist 1 (length l) l))
\end{verbatim}}

{\bf Note:} The {\WOOL} parser ignores extraneous closing parentheses,
allowing you to close top-level expressions in a file by a handful of
closing parentheses, such as:

{\exemplefont\begin{verbatim}
        (defun cons (e l) (+ (list e) l)))))))))))))
        (setq x 1))))))))
\end{verbatim}}

\ITEMb{()}{nil}{the nil value}

{\usagefont\begin{verbatim}
()
\end{verbatim}}\usageupspace

The nil value, backbone of every Lisp implementation. In Lisp an object is
said to be true if it is not nil.

\ITEMa{""}{string notation}

{\usagefont\begin{verbatim}
"string"
\end{verbatim}}\usageupspace

Strings are surrounded by double quotes. C-style escape sequences are
allowed, that is: 

\begin{center}\begin{tabular}{ll}
{\bf Sequence} & {\bf stands for} \\ \hline
\verb|\n| & newline (control-J)\\
\verb|\r| & carriage return (control-M)\\
\verb|\t| & tab (control-I)\\
\verb|\e| & escape (control-[)\\
\verb|\\| & the backslash character itself \verb|\|\\
\verb|\"| & double quote\\
\verb|\xNN| & the character of ascii code NN in hexadecimal\\ 
\verb|\nnn| & the character of ascii code nnn in octal\\
\verb|\c| & c if c is any other character.\\
\end{tabular} \end{center} 

Moreover, you can ignore end of lines in strings by prefixing them with
\verb"\", as in:

{\exemplefont\begin{verbatim}
        (print "This is a very \
        long string")            ==> This is a very long string
\end{verbatim}}

\ITEMb{'}{quote}{prevents evaluation}
        
{\usagefont\begin{verbatim}
'object
(quote object)
\end{verbatim}}\usageupspace

Returns the object {\bf without evaluating} it. The form \verb"'foo" is
not expanded into \verb"(quote foo)" during parsing, but in a
``quoted-expression'' {\WOOL} type for efficiency.

\ITEMci{\%}{*}{/}{arithmetic operators}
        
{\usagefont\begin{verbatim}
(% n1 n2)
(* n1 n2)
(/ n1 n2)
\end{verbatim}}\usageupspace

Returns respectively the modulo, product, and quotient of the integer
arguments as a integer.

\ITEMa{+}{adds or concatenates}
        
{\usagefont\begin{verbatim}
(+ n1 n2 ... nN)
(+ string1 string2 ... stringN)
(+ list1 list2 ... listN)
\end{verbatim}}\usageupspace

Numerically add numbers, concatenate strings or, concatenate lists.
Determines the type of the result as being the type of the first argument
(\verb"()" is a list, \verb|""| is a string, \verb|0| is a number).

\ITEMa{-}{arithmetic difference}
        
{\usagefont\begin{verbatim}
(- n)
(- n1 n2 ...)
\end{verbatim}}\usageupspace

Returns the arithmetic opposition or difference of numbers. 
\verb"(- n1 n2 n3 n4)" is equivalent to \verb"(- n1 (+ n2 n3 n4))".

\ITEMb{=}{equal}{tests equality of any two objects}
        
{\usagefont\begin{verbatim}
(= obj1 obj2)
(equal obj1 obj2)
\end{verbatim}}\usageupspace

Returns obj1 if obj1 is the same as obj2, nil otherwise. This is the
traditional {\bf equal} predicate of Lisp.

Equality of lists is tested by testing the equality of all their elements.

\ITEMa{<}{tests for strict inferiority}
        
{\usagefont\begin{verbatim}
(< n1 n2)
(< string1 string2)
\end{verbatim}}\usageupspace

Compares two numbers or two strings and returns \verb"t" if argument 1 is
inferior and not equal to argument 2 and \verb"nil" otherwise. Strings are
compared alphabetically.

\ITEMa{>}{tests for strict superiority}
        
{\usagefont\begin{verbatim}
(> n1 n2)
(> string1 string2)
\end{verbatim}}\usageupspace

Compares two numbers or two strings and returns \verb"t" if argument 1 is
superior and not equal to argument 2 and \verb"nil" otherwise. Strings are
compared alphabetically.

\ITEMb{?}{print}{prints {\WOOL} objects}
        
{\usagefont\begin{verbatim}
(? obj1 obj2 ... objN)
\end{verbatim}}\usageupspace

Prints the objects, without adding spaces or newlines. Output is flushed at
the end of every call to this function. For now, output goes to the stdout
stream.

\ITEMa{active-label-make}{makes a label (text in a given font)}
        
{\usagefont\begin{verbatim}
(active-label-make label [font])
\end{verbatim}}\usageupspace

Creates a label with string \verb|label| drawn
in the font \verb"font". The text will be redrawn on each expose (The
active label can also be used to paint a string on a pixmap, see
\seep{pixmap-make}).

\context{
foreground &    color of the text string\\
font &          font of the string if not given\\
}

\ITEMa{alone}{specifies that no modifier key is used}
        
\usagetype{Constant}

Specifies that no modifier is pressed for a button or key event for use in
event descriptions.

\ITEMa{and}{logical AND of expressions}
        
{\usagefont\begin{verbatim}
(and obj1 obj2 ... objN)
\end{verbatim}}\usageupspace

Returns \verb"()" as soon as an argument is false (\verb"nil"),
\verb"t" otherwise.

\ITEMa{any}{matches any modifier or button}
        
\usagetype{Constant}

Matches any modifier or button or key in events. It can be also used in many
other functions with appropraite semantics.

\ITEMa{atoi}{ascii string to integer conversion}
        
{\usagefont\begin{verbatim}
(atoi string)
\end{verbatim}}\usageupspace

Returns the integer described by string (in base 10).

{Example:\exemplefont\upspace\begin{verbatim}
        (atoi "123")                    ==>     123
\end{verbatim}}

\ITEMa{atom}{makes an atom from a string}
        
{\usagefont\begin{verbatim}
(atom string)
\end{verbatim}}\usageupspace

Returns the atom of name string. Useful to create atoms with special
imbedded characters, such as \verb"'" or blanks. The inverse function is not
necessary since every {\WOOL} function expecting strings in arguments can
accept atoms instead, and will take the string of their name.  

This should be used when accessing property lists via the \verb|#| function.
For instance if you store properties on machine names (which are strings),
you would use the following call to retrieve the color of a xterm for a
machine:

{Examples:\exemplefont\upspace\begin{verbatim}
        (# (atom machine-name) '(Maalea "Green" Honolua "Blue"))
\end{verbatim}}

\ITEMa{background}{color of the background}
        
\usagetyped{Numeric variable --- screen-relative}{color}

The value of this global variable is used by many functions as the color of
the background. It is a pixel value, such as returned by \verb"color-make",
and is initially set to the pixel of the color \verb"White".  Note that
a non-nil \verb"tile" context value overrides the background value.

\ITEMa{bar-make}{makes a bar descriptor}
        
{\usagefont\begin{verbatim}
(bar-make plug1 plug2 ... plugN)
\end{verbatim}}\usageupspace

Creates a bar of transversal width\footnote{Bars have their own width and a
length which is adjusted to fit around the client window. For an horizontal
bar, the width is their physical height and the length their physical width,
and for a vertical one, it is the opposite} bounded by the current value
of \verb"bar-min-width" and \verb"bar-max-width", and containing the {\tt N}
plugs, which are centered in the bar.  If a plug is \verb"()", it is
considered extensible space, which is expanded when the bar stretches to its
actual dimensions.  The plugs are clipped on the right if necessary.

\context{
fsm & the finite state machine of the bar \\
borderwidth & the width of the bar's border in pixels \\
borderpixel &  the color of the border \\
bordertile & the pixmap used as the border pattern \\
background & the color of the background \\
plug-separator & the number of pixels between consecutive plugs \\
tile & the pattern of its background (may be ()) \\
menu & the default pop-up associated with it \\
cursor & the cursor's shape when the cursor is in it \\
property & the {\WOOL} property associated with the bar \\
bar-min-width & the minimum width in pixels of the bar \\
bar-max-width & the maximun width in pixels of the bar (width for vertical 
bars, height for horizontal bars)\\
}

Note that the plugs are evaluated a second time when the bar is physically
created, allowing you to give expressions for plugs (quoted to evade the
first evaluation of arguments of the \verb"bar-make" function itself)
will evaluate to a plug on the realisation of the wob.  

For instance, to have the window name in a plug, the bar can be made by:
{\exemplefont\begin{verbatim}
        (bar-make 10 20 '(plug-make (label-make window-name)))
\end{verbatim}}
If you don't quote the plug, all the bars will have the same name on the
realisation of the bar, the value of \verb"window-name" at the time of the
call to bar-make.

\ITEMb{bar-min-width}{bar-max-width}{limits to the transversal width of a bar}

\usagetyped{Numeric variable}{number}

The values of these global variables are used by the constructors of bars to
limit their transversal width, clipping the plugs if necessary.
\verb"bar-min-width" defaults to 1 and \verb"bar-max-width" defaults to 100.

\ITEMa{bar-separator}{number of pixels between consecutive bars in menus}
        
\usagetyped{Numeric variable}{number}

The value of this global variable is used by the constructors of menus to
yield the number of pixels separating two contiguous bars.  Default is 4
pixels. Used in  \verb"menu-make" function.

\ITEMa{bell}{rings the keyboard bell}

{\usagefont\begin{verbatim}
(bell [percent])
\end{verbatim}}\usageupspace

Rings the bell on the keyboard, with volume set by the \verb"percent"
numeric argument, which can take values ranging from -100 to 100 (defaults
to 0). -100 means lowest volume, 0 means base volume, 100 means full volume.

\ITEMc{bitwise-and}{bitwise-or}{bitwise-xor}{bitwise operators}
        
{\usagefont\begin{verbatim}
(bitwise-op n1 n2 ... nN)
\end{verbatim}}\usageupspace

Returns the result of the bitwise operator on the N arguments.

{Examples:\exemplefont\upspace\begin{verbatim}
        (bitwise-or 1 4)                ==>     5
        (bitwise-and 3 5)               ==>     1
        (bitwise-xor 3 5)               ==>     6
\end{verbatim}}

\ITEMa{borderpixel}{color of the border of a wob}
        
\usagetyped{Numeric variable --- screen relative}{color}

The value of this global variable is used by the constructors of all wobs as
the color of their border. It is a pixel value, such as returned by
\verb"color-make", and is initially set to the pixel of the color
\verb"Black". It is always overridden by the value of \verb"bordertile" if
not nil.

\ITEMa{bordertile}{pixmap to tile the border of a wob}
        
\usagetyped{Variable}{pixmap}

The value of this global variable is used by the constructors of all wobs as
the pixmap to display in their border. It is a pixmap object, such as
returned by \verb"pixmap-make". If set to (), the borderpixel value is used.

\ITEMa{borderwidth}{width in pixels of the border of a wob}
        
\usagetyped{Numeric variable}{number}

The value of this global variable is used by the constructors of all wobs to
set their border width in pixels. A value of 0 means no border.

\ITEMa{boundp}{tests if an atom has already been defined}
        
{\usagefont\begin{verbatim}
(boundp atom)
\end{verbatim}}\usageupspace

Returns the (evaluated) atom if it has been defined, () otherwise.

{Examples:\exemplefont\upspace\begin{verbatim}
        (setq foo 1)
        (boundp 'foo)                   ==>     foo
        (boundp 'bar)                   ==>     ()
\end{verbatim}}

\ITEMa{button}{makes a button event}
        
{\usagefont\begin{verbatim}
(button number modifier)
\end{verbatim}}\usageupspace

Returns a button event matching the X event {\bf ButtonPress} of button
number \verb"number", and will verify that
the modifier argument was depressed during the click. This event will wait
for the corresponding {\bf ButtonRelease} event before returning.  

Number or modifier can take the value \verb"any", thus matching any of the
values of the corresponding argument.  These button events can then be used
in the matching part of the transitions of fsms or in the \verb"grabs"
field of a window-make description.

\exemples{Examples:}{
(button 1 with-alt) & matches clicking left button of the mouse
                                while keeping the alt key depressed.\\
(button any with-control) & matches control-clicking with any button\\
}

\ITEMa{buttonpress}{makes a buttonpress event}
        
{\usagefont\begin{verbatim}
(buttonpress number modifier)
\end{verbatim}}\usageupspace

Same as button, but does not wait for the buttonrelease, meaning that the
buttonrelease is available for another transition.
This can be used to trigger \verb"move-window", since this function waits for
a button release to terminate.

\ITEMa{buttonrelease}{makes a buttonrelease event}
        
{\usagefont\begin{verbatim}
(buttonrelease number modifier)
\end{verbatim}}\usageupspace

Same as button, but only matches a button release. Useful when you cannot
wait for a button event, for instance when iconifying a window, because if
the window is iconified on a button press, the correponding release event
will be lost (in fact will go to the window underneath)

\ITEMa{check-input-focus-flag}{follow input hint for setting focus}
        
\usagetyped{Numeric variable}{number}

If this flag is set to {\bf 1}, which is the default, {\GWM} will refuse to
set the keyboard focus via the \verb"set-focus" call to windows having
declared that they didn't need the focus. This flag is provided so that you
can use clients which set their input hint the wrong way.

\ITEMb{circulate-windows-down}{circulate-windows-up}{circulates mapped windows}
        
{\usagefont\begin{verbatim}
(circulate-windows-up)
(circulate-windows-down)
\end{verbatim}}\usageupspace

Put the topmost window down, or the lowest window up respectively.

\ITEMa{color-make}{finds a pixel color by name}
        
{\usagefont\begin{verbatim}
(color-make color-name)
\end{verbatim}}\usageupspace

Takes a string \verb"color-name" describing the color by its english  name
(as found in the \verb"rgb.txt" system file),
and returns the pixel (number) associated with the closest existing color on
the screen.  This pixel can later be used as values for foreground and
background, or in the \verb"pixmap-make" function.  If color is not found,
returns the closest allocated color. Color is in fact the number of the
pixel value, which is the index of the closest color in the color table.

Colors can also be specified directly by their {\bf RGB}\footnote{Red,
Green, Blue.} values, with the {\bf\#}-convention: if the \verb"color-name"
string begins with a sharp sign, the following characters are taken as
hexadecimal digits specifying the most significant part of the Red, Green
and, Blue values, with the following syntax: (each letter stands for a digit)

\begin{verbatim}
#RGB
#RRGGBB
#RRRGGGBBB
#RRRRGGGGBBBB
\end{verbatim}

Example:{\exemplefont\upspace\begin{verbatim}
        (color-make "DarkSlateBlue")
        (color-make "light grey")
        (color-make "#f00")             ; pixel for red
        (color-make "#300000a076be")
\end{verbatim}}

\ITEMa{compare}{ordering comparison}

{\usagefont\begin{verbatim}
(compare n1 n2)
(compare string1 string2)
\end{verbatim}}\usageupspace

Compares two numbers or two strings and returns \verb"-1", \verb"0" or
\verb"1" if its first argument is lesser than, equal to or, greater than the
second.

\ITEMa{cond}{conditional test}
        
{\usagefont\begin{verbatim}
(cond (condition then) [(condition then) ...])
\end{verbatim}}\usageupspace

This is the classical ``cond'' Lisp function, returning the evaluation of the
``then'' part of the first true condition.

{\exemplefont\begin{verbatim}
        (defun fib (n)
            (cond
                ((= n 0) 1)
                ((= n 1) 1)
                (t (+ (fib (- n 1)) 
                        (fib (- n 2))))))
\end{verbatim}}

\ITEMa{confine-grabs}{cursor stays confined in grabbing wobs}

\usagetyped{Numeric variable}{number}

If set, during all grabs (either via \verb"pop-menu" or \verb"grab-server")
will confine the pointer inside the grabbing wob during the duration of
the grab.

\ITEMb{context-save}{context-restore}{context management}
        
{\usagefont\begin{verbatim}
(context-save context)
(context-restore context)
\end{verbatim}}\usageupspace

A context is a kind of property list, it is an even-sized list whose even
elements are atoms and whose odd ones are their values (see the \see{with}
function), of the form {\tt (var{\it 1} val{\it 1} var{\it 2} val{\it 2} ...
var{\it N} val{\it N})}. \verb"context-save" creates a new context
where the {\tt val{\it i}} are the result of the evaluation of the
{\tt var{\it i}} in the argument context, whereas context-restore does
a {\tt (setq var{\it i} val{\it i})} for each of its variable/value pairs.

{\bf Note:} the provided {\tt val{\it i}} serves as default for
\verb"context-save" in the case where the corresponding {\tt var{\it i}} is
undefined:

{\exemplefont\begin{verbatim}
        (setq my-context (list 'a 1 'b 2))
        (setq a "foo")                     ; b is undefined
        (print (context-save my-context))
                                        ==>  (a foo b 2)
\end{verbatim}}

\ITEMa{copy}{copies a wool object\hfill{\bf EXPERT}}

{\usagefont\begin{verbatim}
(copy object)
\end{verbatim}}\usageupspace

Returns a copy of the \verb"object" argument, which can only be a list for
now. This function is flagged as ``{\em Expert}\,'' because it is of use
to people doing physical replacement functions, which are reserved to
experts.

\ITEMa{current-event-code}{code of the last event}
        
{\usagefont\begin{verbatim}
(current-event-code)
\end{verbatim}}\usageupspace

Returns the code (button or keycode) of the {\bf last} event received (The
one which triggered the transition you are in).

\ITEMa{current-event-from-grab}{tests if last event was generated by a grab}
        
{\usagefont\begin{verbatim}
(current-event-from-grab)
\end{verbatim}}\usageupspace

If the last event was a crossing or focus event consecutive to a grab set or
removed, returns \verb"t".

\ITEMa{current-event-modifier}{modifier of the last event}
        
{\usagefont\begin{verbatim}
(current-event-modifier)
\end{verbatim}}\usageupspace

Returns the modifier (state of shift, control, alternate,...) keys that were
depressed when the {\bf last} event occurred (The one which triggered the
transition you are in).  Note that the combination of two modifiers is
expressed by bitwise-{\bf or}ing the modifiers.

\ITEMa{current-event-time}{time in milliseconds of the last event}

{\usagefont\begin{verbatim}
(current-event-window-coords)
\end{verbatim}}\usageupspace

Returns the time at which occured the last button, key, crossing, or focus 
event. Time is expressed as a number of milliseconds.

\ITEMa{current-event-window-coords}{relative position of the last event}
        
{\usagefont\begin{verbatim}
(current-event-window-coords)
\end{verbatim}}\usageupspace

Returns the list of coordinates of the last event, if it was a
pointer or a key event, relative to the current client window.

This list has six elements:
\desctable{Element}{Meaning}{
0       & x position in size increments \\
1 & y position in size increments \\
\ & \multicolumn{1}{r}{\small\it (character positions for xterm)} \\
2 & same x position in pixels\\
3 & same y position in pixels\\
4 & x position in pixels relative to the decorating window\\
5 & y position in pixels relative to the decorating window\\
}

{\bf WARNING:} The position in size increments do not work in the general
case, but only for windows having the text widget flush to the bottom right
corner of their main windows, like Xterm. There is no fix for it, but this
function is a kind of a hack anyways.

\ITEMd{current-event-x}{current-event-y}{current-event-relative-x}{current-event-relative-y}{position of the last event }
        
{\usagefont\begin{verbatim}
(current-event-x)
(current-event-y)
(current-event-relative-x)
(current-event-relative-y)
\end{verbatim}}\usageupspace

Returns the x or y coordinate (number)  of the mouse during the {\bf last}
event, if it was a key or button event (The one which triggered the
transition you are in).

The first coordinates are relative to the root, whereas the last ones are
relative to the wob where they occurred.

\ITEMa{current-mouse-position}{queries server for mouse position}
        
{\usagefont\begin{verbatim}
(current-mouse-position)
\end{verbatim}}\usageupspace

Queries the server for the mouse state and returns it as a list of
four element where the first element is x, second is y, third is
state of modifiers and buttons bitwise-{\bf or}ed, and fourth
is the number of the screen where the pointer is.

\ITEMa{current-user-event}{name of the last user event}

{\usagefont\begin{verbatim}
(current-user-event)
\end{verbatim}}\usageupspace

If the last event was an user event, returns the label of the event (the
atom that was given as argument to the \verb"send-user-event" call).
Triggers an error if the last event was not a user event.

\ITEMa{cursor}{shape of the cursor in a wob}
        
\usagetyped{Variable}{cursor}

The value of this global variable, which must be a cursor returned by a call
to \verb"cursor-make", is used by the constructors of all wobs to set the
appearance of the mouse cursor when the pointer is in the wob. If set to (),
the cursor will not change when entering the wob.

This is also used by functions like \verb"grab-server", \verb"move-window",
and \verb"resize-window" to change the shape of the cursor during a function.

\ITEMa{cursor-NW}{cursor shapes for the 8 directions}

\usagetyped{Variables}{cursor}

These values ({\tt cursor-NW, cursor-NE, cursor-N, cursor-SW, cursor-SE,
cursor-S, cursor-E, cursor-W}) defines cursors to be used on the eight
directions by some functions, such as resize-window with the mwm-like style
(\seep{resize-style}). The eigth corners are respectively: NorthWest,
NorthEast, North, SouthWest, SouthEast, South, East and West.

\ITEMa{cursor-make}{makes a cursor with a bitmap and a mask}
        
{\usagefont\begin{verbatim}
(cursor-make foreground-bitmap-filename 
             mask-bitmap-filename)
(cursor-make cursor-name)
(cursor-make number)
\end{verbatim}}\usageupspace

Constructs a mouse cursor with two bitmaps (strings containing the file
names). This cursor can then be used in the cursor variable. The bitmaps are
files searched for in the same path as in the \see{pixmap-make} function.

\sloppy The convention in use for naming a cursor \verb"foo" is to name the
foreground bitmap as \verb"foo-f.xbm" and the mask bitmap as
\verb"foo-m.xbm". This convention is used in the calling of
\verb"cursor-make" with one argument, so that \verb|(cursor-make "foo")| is
equivalent to \verb|(cursor-make "foo-f" "foo-m")|.

You can also define the cursor as one of the predfined ones in the server
cursor font by giving its index in the font as the \verb"number" argument to
cursor-make. See the Appendix B of the {\bf Xlib} manual for the list of
available cursors. For instance, ``star trek'' cursor can be made by the
call \verb"(cursor-make 142)".
The symbolic names of these cursors are defined in the \verb"cursor-names.gwm"
file. Once this file is loaded, you can for instance use the ``star trek''
cursor by a \verb"(cursor-make XC_trek)".

\ITEMa{cut-buffer}{contents of cut buffer 0}
        
\usagetyped{Active value}{string}

Its value is the content of the X cut buffer 0, returned as a string.  When
set, it takes the string argument and stores it in the same cut buffer. This
can be used to communicate with clients still using this obsolete way to
perform cut-and-paste operations, such as xterm. 

\ITEMf{defun}{defunq}{lambda}{lambdaq}{de}{df}{defines a {\WOOL} function}
        
{\usagefont\begin{verbatim}
(defun function-name (arg1 ... argn) instructions ...)
(defunq function-name (arg1 ... argn) instructions ...)
(lambda (arg1 arg2 ... argn) instructions ...)
(lambdaq (arg1 arg2 ... argn) instructions ...)
\end{verbatim}}\usageupspace

Defines a Lisp function and returns the atom pointing to the function. The
list of arguments must be present, even if ().  If defined by \verb"defun",
the function will evaluate its arguments before executing, and will not
evaluate them if defined by \verb"defunq". The return value of an execution
of the function is the value returned by the last instruction of its body.
\verb"de" and \verb"df" are just synonyms for \verb"defun" and
\verb"defunq". The \verb"lambda" call, which evaluates its arguments,
defines a function (without binding it to a name) while \verb"lambdaq"
creates a function which does not evaluates its arguments. The two following
expressions are equivalent:

{\exemplefont\begin{verbatim} 
        (defun foo(x) (+ x 1))
        (setq foo (lambda (x) (+ x 1)))
\end{verbatim}}

When lists are evaluated, {\WOOL} applies the result of the evaluation of
the CAR of the list to its CDR, like in the {\bf Scheme} language. Thus to
apply a lambda function to its arguments, just eval the list constructed
with the lambda construct and its arguments. The classic Lisp {\tt apply}
function could thus be defined by:

{\exemplefont\begin{verbatim}
        (defun apply (function list-of-arguments)
            (eval (+ (list function) list-of-arguments)))
\end{verbatim}}

Since functions are Lisp objects, to define a synonym of a function, you must
use \verb"setq". Thus, you can change the meaning of a function easily, for
instance to make \verb"move-window" always raise the window, you would say
in your profile:

{\exemplefont\begin{verbatim}
        (setq original-move-window move-window)
        (defun move-window () 
                (raise-window)
                (original-move-window))
\end{verbatim}}

This means also that an atom can only have one value, and {\tt (setq
move-window 1)} will erase the definition of move-window if you didn't save
it in another atom.

Example:{\exemplefont\upspace\begin{verbatim}
        (defunq incr (value delta)
                (set value (+ (eval value) (eval delta))))

        (setq x 4)
        (incr x 2)
        (print x)                       ==>     6
\end{verbatim}}

Functions taking a variable number of arguments can be defined by
providing a parameter name instead of a list of parameters. In the
body of the function during its execution, this parameter will be set
to the list of the parameters given to the function.

{\exemplefont\begin{verbatim}
        (defun max l
            (with (max-val 0) 
                (for obj l (if (> obj max-w)
                                  (setq max-w obj)))
                max-w))
        (max)                           ==>     0
        (max 34 65 34 12)               ==>    65
\end{verbatim}}

{\bf Note:} You are not allowed to redefine active values (such as
\verb"window") or numeric variables (such as \verb"foreground"), or use
them as names for the parameters to the function.

{\exemplefont\begin{verbatim}
        (defun window (a b) (+ a b))    ==>     ERROR!
        (defun my-window (wob) 
            (window wob))               ==>     ERROR!
\end{verbatim}}

\ITEMa{delete-nth}{physically removes an element of a list\hfill{\bf EXPERT}}

{\usagefont\begin{verbatim}
(delete-nth index list)
(delete-nth key list)
\end{verbatim}}\usageupspace

This function physically removes an element of a list and returns the list. 
Elements are specified as for the \verb"#" function. 
For property lists, the key and the value are deleted.

{\exemplefont\begin{verbatim}
        (setq l '(a 1 b 2 c 3 d 4))
        (delete-nth 5 l)
        l                               ==>     (a 1 b 2 c d 4)
        (delete-nth 'b l)
        l                               ==>     (a 1 c d 4)
\end{verbatim}}

\ITEMa{delete-window}{asks client to delete one of its windows}

{\usagefont\begin{verbatim}
(delete-window [window])
\end{verbatim}}\usageupspace

This function will ask the client owning the argument \verb"window" (or the
current window if none is given) to delete it. This is only possible if the
client participates in the \verb"WM_DELETE_WINDOW" ICCC protocol, in which
case the function returns \verb"t", otherwise \verb"()".

This is different from the \verb"kill-window" call, which destroys the
client and every top-level window owned by it, in that other windows should
survive the call.

\ITEMc{dimensions}{width}{height}{dimensions of a {\WOOL} object}

{\usagefont\begin{verbatim}
(dimensions object)
(width object)
(height object)
\end{verbatim}}\usageupspace

These functions return the width and height in pixels of any {\bf wob}, {\bf
pixmap}, {\bf active label}, {\bf cursor}, or {\bf string}.  For wobs, the
returned dimensions {\bf include} the borderwidth. For strings, these are the
dimensions that such a string would take on the screen, if printed in the
current {\tt font}, thus including the \verb"label-horizontal-margin" and
\verb"label-vertical-margin".

The \verb"dimensions" function returns a list of {\bf four} values: x, y,
width, heigth. x and y being non-null only for wobs in which case they are
the coordinates of the upper-left corner of the wob relative to the parent
wob.

\ITEMa{direction}{direction of menus}

\usagetyped{Numeric variable}{number}

This variable controls the main direction of menus created by the
\see{menu-make} constructor. Possible values are \verb"horizontal"
or \verb"vertical".

\ITEMa{describe-screen}{user function called to describe a screen}
        
{\usagefont\begin{verbatim}
(describe-screen)
\end{verbatim}}\usageupspace

This function {\bf must} be provided by the user. 
It must return the description of the screen (cursor, fsm, ...), made with a
window-make call. Of the window-make parameters, only the \verb"opening"
(expression evaluated when all windows are decorated, just before entering
the main loop of {\GWM}) and \verb"closing" (expression evaluated just before
ending) parameters are used. 

\context{
  fsm  &  for the fsm of the root window \\
  menu &  for the default menu of the root window \\
  cursor &  for the shape of the cursor when in root \\
  property &  for the initial value of the property field \\
  tile &  for the pixmap to be tiled on the screen
    (() means do not touch the exixting screen) \\
}

This function is called by {\GWM} for each managed screen once. In the
current version, only one screen is managed. Before each invocation of this
function, the screen characteristics (visual, depth, ...) are updated to
reflect the characteristics of the corresponding screen.

\ITEMa{describe-window}{user function called to decorate a new window}
        
{\usagefont\begin{verbatim}
(describe-window)
\end{verbatim}}\usageupspace

When a new window must be decorated by {\GWM}, it calls this user-defined
function with \verb"window" and \verb"wob" set to the new window. It must
return a list of two window descriptors made with \see{window-make}
describing respectively the window and the icon of the new client window.

A recommended way to use \verb"describe-window"  is to store in advance all
the window descriptions in the resource manager by \see{resource-put} calls
and use a \see{resource-get} call in your \verb"describe-window" function to
retrieve the appropriate description. But this is not mandatory and you are
free to design the way to describe a window freely.  For instance with the
following describe-window function:

{\exemplefont\begin{verbatim}
        (defun describe-window ()
            (resource-get (+ window-client-class "." window-client-name)
                    "any-client.any-client"))

        (resource-put "any-client.any-client" any-window)
        (resource-put "XTerm.xterm" xterm-window)
\end{verbatim}}

{\GWM} will decorate xterm windows by the description held in
\verb"xterm-window", and will decorate the windows of any other client by
the \verb"any-window" description.

\ITEMa{display-name}{name of the X server}
        
\usagetyped{Variable}{string}

This variable holds the name of the display on which {\GWM} is running.

\ITEMb{double-button}{double-buttonpress}{makes a double-click button event}
        
{\usagefont\begin{verbatim}
(double-button number modifier)
(double-buttonpress number modifier)
\end{verbatim}}\usageupspace

These events look for an incoming X {\bf ButtonPress} event and matches it
if the {\bf last} key or button event was on the same button, in the same
wob, and not a key event, and happened less than \verb"double-click-delay"
milliseconds before it.

The button to be pressed is specified by the number \verb"number", and these
events verify that the modifier argument was depressed during the click.
\verb"double-button" waits for the corresponding {\bf ButtonRelease}
event before returning, while \verb"double-buttonpress" returns
immediately.  Number or modifier can take the value \verb"any", thus
matching any of the values of the corresponding argument.  

Note that handling double-click events this way implies that the action that
is done on a simple click event is executed even for a double click event,
just before the action associated with it. The recommended way to use
double-clicks is to place the \verb"double-button" event {\bf before} the
\verb"button" in the state of the fsm as in:

{\exemplefont\begin{verbatim}
        (on (double-button 1 any) (do-double-action))
        (on (button 1 any) (do-simple-action))
\end{verbatim}}

{\bf Note:} The ``last'' event can be a {\bf ButtonRelease} event only if it
is waited for explicitly by a {\GWM} \verb"buttonrelease" event, in which
case the \verb"double-click-delay" is measured between the release of the
first click and the press of the second. Otherwise (if the first click is
awaited by a \verb"button" event) the delay is taken between the two
``presses'' of the button.

\ITEMa{double-click-delay}{maximum time between double clicks}

\usagetype{Numeric variable}

If two buttonpress events are closer in time than \verb"double-click-delay"
milliseconds then the second one can be matched by the \verb"double-button"
and \verb"double-buttonpress" events.

\ITEMa{draw-line}{draws a line in a pixmap}
        
{\usagefont\begin{verbatim}
(draw-line pixmap x1 y1 x2 y2)
\end{verbatim}}\usageupspace

Draws a line (via the {\bf XDrawLine} X call) in a pixmap from point
\verb"x1", \verb"y1" to point \verb"x2", \verb"y2" in the pixmap
coordinates. 

\contextdim{-2cm}{
foreground & color of the line \\
}

\ITEMa{elapsed-time}{gets time used by {\GWM}}

{\usagefont\begin{verbatim}
(elapsed-time)
\end{verbatim}}\usageupspace

Returns the time in 60ths of a second for which {\GWM} has been running. It
is a CPU time for benchmarking, not a real one.

\ITEMa{end}{terminates {\GWM}}
        
{\usagefont\begin{verbatim}
(end)
\end{verbatim}}\usageupspace

Terminates {\GWM}, de-iconifying all the windows, un-decorating them,
restoring their orignal borderwidth, and closing the display.

\ITEMb{enter-window}{leave-window}{events generated when the pointer crosses the border of a wob}
        
\usagetyped{Constant}{event}

This event is sent to the wob when the pointer crosses its border.
Note that no leave event is generated if the pointer goes over a child window
inside it.

\ITEMa{eq}{tests strict equality of any two objects\hfill{\bf EXPERT}}

{\usagefont\begin{verbatim}
(eq object1 object2)
\end{verbatim}}\usageupspace

Returns true only if the two object are the same, i.e., if they are at the
same memory location.

Example:{\exemplefont\upspace\begin{verbatim}
        (setq l '(a b c))
        (setq lc '(a b c))
        (= l lc)                        ==>        t
        (eq l lc)                       ==>        ()
        (eq (# 1 l) (# 1 lc))           ==>        t
\end{verbatim}}

\ITEMa{error-occurred}{traps errors occurring in expressions}

{\usagefont\begin{verbatim}
(error-occurred expressions ...)
\end{verbatim}}\usageupspace

Executes the given expressions and returns \verb"()" if everything went
fine, but returns \verb"t" as soon as a {\WOOL} error occurred and does not
print the resulting error message.

\ITEMa{eval}{evaluates a {\WOOL} expression}
        
{\usagefont\begin{verbatim}
(eval wool-expression)
\end{verbatim}}\usageupspace

This function evaluates its argument. Note that there is a double evaluation,
since the argument is evaluated by Lisp before being evaluated by eval.

Example:{\exemplefont\upspace\begin{verbatim}
        (? (eval (+ '( + ) '(1 2))))     ==>     prints "3"
\end{verbatim}}

\ITEMa{execute-string}{executes a {\WOOL} string}
        
{\usagefont\begin{verbatim}
(execute-string string)
\end{verbatim}}\usageupspace

Parses the string argument as a {\WOOL} program text and evaluates the read
expressions. Returns \verb"()" if an error occurred in the evaluation,
\verb"t" otherwise. Very useful in conjunction with \see{cut-buffer},
to evaluate the current selection with:

{\exemplefont\begin{verbatim}
        (execute-string cut-buffer)
\end{verbatim}}

\ITEMb{focus-in}{focus-out}{events received when input focus changes on the client window}
        
\usagetyped{Constant}{event}

These events are sent to a wob when a child receives or loses the keyboard
focus, i.e., the fact that all input from the keyboard goes to that child
regardless of the pointer position. Only window wobs receive these events
when their client window gets the focus,
they must redispatch them to their sons (bar wobs) if they are expected to
respond to such events (a title bar might want to invert itself for instance)
by using the \verb"send-user-event" function.

\ITEMa{font}{default font}
        
\usagetyped{Numeric variable}{number}

This is the font ID used by default for the
constructors \verb"label-make" and \verb"active-label-make". Initialized to
``fixed'' or your local implementor's choice.
It is also the font returned by \verb"font-make" when it fails to find a font.

\ITEMa{font-make}{loads a font}
        
{\usagefont\begin{verbatim}
(font-make font-name)
\end{verbatim}}\usageupspace

This function loads a font in memory, from the ones available on the server.
You can list the available fonts by the XLSFONTS(1) UNIX command. The function
returns the descriptor (number) of the loaded font. If it fails to find it,
it will issue a warning and return the default font found in the \verb"font"
global variable.

\ITEMb{for}{mapfor}{iterates through a list of values}
        
{\usagefont\begin{verbatim}
(for variable list-of-values inst1 inst2 ... instN)
(mapfor variable list-of-values inst1 inst2 ... instN)
\end{verbatim}}\usageupspace

This Lisp control structure affects successively the variable (not
evaluated) to each of the elements  of the evaluated list-of-values, and
executes the n instructions of the body of the for.  The variable is local
to the for body and will be reset to its previous value on exit.

\verb"for"  returns the value of the evaluation on \verb"instN" in
the last iteration, whereas \verb"mapfor" builds a list having for
values the successive values of \verb"instN" for the iterations.

Examples:{\exemplefont\upspace\begin{verbatim}
        (for window (list-of-windows)           ; will replace all windows
            (move-window 0 0))                  ; in the upper-left corner

        (for i '(a b 2 (1 2)) 
            (print i ","))              ==>     a,b,2,(1 2),
        (mapfor i '(a b 2 (1 2))
            (list i))                   ==>     ((a) (b) (2) ((1 2)))
\end{verbatim}}

\ITEMa{foreground}{color of the foreground}
        
\usagetyped{Numeric variable --- screen relative}{color}

The value of this global variable is used by the constructors of graphics
(labels and pixmaps) to paint the graphic. It is a pixel value, such as
returned by \verb"color-make", and is initially set to the pixel of the color
\verb"Black".

\ITEMa{freeze-server}{stops processing other clients during grabs}
        
\usagetyped{Numeric variable}{number}

If set to \verb"t" or 1, the X server is {\bf frozen}, i.e. it doesn't
process the requests for other clients during the \verb"move-window",
\verb"resize-window" and \verb"grab-server" operations, so that you cannot
have your pop-up menu masked by a newly mapped window, or you display
mangled when moving or resizing a window.

If set to () or 0, the processing of other clients requests are allowed so
that you can print in an xterm while popping a menu, for instance. (If the
server is frozen, {\GWM} tries to write on an xterm whose X output is blocked
and {\GWM} is deadlocked, forcing you to log on another system to kill it!).
This should happen only when debugging, though, so that the default for this
variable is \verb"t" (Menus are normally coded so that user actions are
triggered {\bf after} de-popping).

\ITEMa{fsm}{Finite State Machine of the wob}
        
\usagetyped{Variable}{fsm}

This global variable is used by all wob constructors to determine their future
behavior. Holds a fsm made with \verb"fsm-make", or \verb"()" (no behaviour).

\ITEMa{fsm-make}{compiles an automaton}
        
{\usagefont\begin{verbatim}
(fsm-make state1 state2 ... stateN)
\end{verbatim}}\usageupspace

This function creates a finite state machine with {\tt N} states. Each wob
has an associated fsm, and a current state, which is the first state at
creation time. Each time a wob receives an event, its current state of its
fsm is searched for a transition which matches the event. If found, the
corresponding action is executed and the current state of the fsm becomes
the destination state of the transition. If no destination state was
specified for the transition (see the \verb"on" function), the current state
of the fsm does not change. See \verb"on" and \verb"state-make" functions.

\ITEMa{geometry-change}{event generated when window changes size}
        
\usagetyped{Constant}{event}

This event is sent to the window when its geometry changes by a resize
operation either from the client or from {\GWM}.

\ITEMa{get-wm-command}{gets the WM\_COMMAND property}

{\usagefont\begin{verbatim}
(get-wm-command)
\end{verbatim}}\usageupspace

Returns the command line that can be used to restart the application
which created the current window in its current state as a list of
strings.  (For instance, \verb|("xterm" "-bg" "Blue")|). See {\tt
save-yourself}, p~\pageref{save-yourself}.

\ITEMa{get-x-default}{gets a server default}

{\usagefont\begin{verbatim}
(get-x-default program-name option-name)
\end{verbatim}}\usageupspace

Calls the ``{\tt XGetDefault}'' Xlib function to
query the server defaults about an option for a program. This function
should be used to retrieve defaults set either by the obsolete {\tt
.Xdefaults} files or the XRDB client. It returns \verb"()" if no such
option was defined, or the option as a {\WOOL} string.

\ITEMa{get-x-property}{gets an X property on the client}
        
{\usagefont\begin{verbatim}
(get-x-property property-name)
\end{verbatim}}\usageupspace

Returns the value of the X11 property of name \verb"property-name" of
the current client window. It only knows how to handle the types STRING
and INTEGER. For instance, {\tt (window-name)} is equivalent to {\tt
(get-x-property "WM\_NAME")}.

\ITEMa{getenv}{gets the value of a shell variable}
        
{\usagefont\begin{verbatim}
(getenv variable-name)
\end{verbatim}}\usageupspace

If variable-name (string) is the name of an exported shell variable, getenv
returns its value as a string, otherwise returns the nil string.

\ITEMa{grab-keyboard-also}{grab-server grabs also keyboard events}

\usagetyped{Numeric variable}{number}

If set, all grabs will also grab the keyboard, i.e. all keyboard events will
also be redirected to the grabbing wob, whereas if unset only the pointer
events get redirected.

\ITEMa{grab-server}{grabs the server}
        
{\usagefont\begin{verbatim}
(grab-server wob ['nochild])
\end{verbatim}}\usageupspace

Grab-server freezes the server, so that only requests from {\GWM} are
processed, and every event received by {\GWM} is sent to the wob (called the
``grabbing'' wob) passed as argument, or to a child of it. Re-grabbing the
server afterwards just change the grabbing wob to the new one if it is not a
child of the wob currently grabbing the server, in which case the function
has no effect. Grabbing the server by a wob is a way to say that all events
happening for wobs which are not sons of the grabbing wob are to be
processed by it.

For instance, the pop-menu function uses \verb"grab-server" with the menu as
argument in order for the sons to be able to receive enter and leave window
events and to catch the release of the button even outside the pop-up.

Grabbing the server sets the mouse cursor to the value of \verb"cursor" if not
().

If the atom \verb"nochild" is given as optional argument, if an event
was sent to a child of the grabbing wob, it is redirected to the grabbing
wob as well (default not to redirect it).

\ITEMa{grid-color}{color to draw (xor) the grids with}
        
\usagetyped{Numeric variable --- screen relative}{color}

This is the color (as returned by color-make) that will be used to draw the
grid with (in XOR mode) on the display. Initially set to the
``Black'' color.

\ITEMa{grabs}{passive grabs on a window}

\usagetyped{Variable}{list}

The events specified in this list are used by the \see{window-make}
constructor to specify on which events {\GWM} establishes passive
grabs on its window to which the client window gets reparented. Passive
grabbing means redirecting some events from a window (bars, plugs, client)
to the main window, ``stealing'' them from the application.

\ITEMa{gwm-quiet}{silent startup}
        
\usagetyped{Numeric variable}{0/1}

This variable evals to 1 if the users specified (by the \verb"-q" option) a
quiet startup. Your code should check for this variable before printing
information messages.

\ITEMa{hack}{raw access to {\GWM} internal structures\hfill{\bf EXPERT}}
        
{\usagefont\begin{verbatim}
(hack type memory-location)
\end{verbatim}}\usageupspace

{\bf WARNING:} Internal debug function: returns the {\WOOL} object constructed
from C data supposed present at memory-location (number). The type of the
type argument is used to know what to find: 

\begin{center}\begin{tabular}{lll}
{\bf type of type}&{\bf interpretation of pointer}&{\bf returns}\\ \hline
 number & pointer on integer & number \\
 string & C string & string \\
 () & object & object \\
 atom & pointer to an {\WOOL} object & number \\
\end{tabular}\end{center}

This function is {\bf not} intended for the normal user, it should be used
only as a temporary way to implement missing functionalities or for
debugging purposes.

\ITEMa{hashinfo}{statistics on atom storage}
        
{\usagefont\begin{verbatim}
(hashinfo)
\end{verbatim}}\usageupspace

Prints statistics on atom storage.

\ITEMb{horizontal}{vertical}{directions of menus}
        
\usagetyped{Constant}{number}

Used to specify orientation of menus in menu-make.

\ITEMa{iconify-window}{iconifies or de-iconifies a window}
        
{\usagefont\begin{verbatim}
(iconify-window)
\end{verbatim}}\usageupspace

Iconify the current window (or de-iconify it if it is already an icon). The
current window is then set to the new window. 

{\bf WARNING:} Never trigger this function with a \verb"button" event, but
with a \verb"buttonpress" or \verb"buttonrelease" event. If you trigger it
with a button event, the window being unmapped cannot receive the
corresponding release event and {\GWM} acts weird!

{\bf WARNING:} Due to the grab management of X, an unmapped window cannot
receive events, so a grab on this window is automatically lost.

\ITEMa{if}{conditional test}
        
{\usagefont\begin{verbatim}
(if condition then [condition then] ... [condition then]
              [else])
\end{verbatim}}\usageupspace

This is similar to \verb"cond" but with a level of parentheses suppressed.  It
executes the ``then'' part of the first true condition, or the else condition
(if present) if no previous condition is true.

{\exemplefont\begin{verbatim}
        (defun fib (n)
            (if (= n 0)
                    1
                (= n 1) 
                    1
                    (+ (fib (- n 1) (- n 2)))))
\end{verbatim}}

\ITEMa{inner-borderwidth}{borderwidth of the client window}
        
\usagetyped{Numeric variable}{number}

When {\GWM} decorates a new window, it sets the width of the border of the
client window to the value of the \verb"inner-borderwidth" variable at the
time of the evaluation of the \verb"window-make" call.

If \verb"inner-borderwidth" has the value \verb"any", which is the default
value, the client window borderwidth is not changed.

\ITEMa{invert-color}{color to invert (xor) the wobs with}
        
\usagetyped{Numeric variable --- screen relative}{color}

This is the color (as returned by color-make) that will be used by 
\seeref{wob-invert} to quickly XOR the wob surface.
Initially set to the bitwise-xoring of Black and White colors for each
screen.

\ITEMa{invert-cursors}{inverts the bitmaps used for making cursors}
        
\usagetyped{Numeric variable}{number}

Due to many problems on different servers, you might try to set to 1 this
numerical global variable if your loaded cursors appear to be video
inverted. Defaults to 0 (no inversion).

\ITEMa{itoa}{integer to ascii string conversion}
        
{\usagefont\begin{verbatim}
(itoa number)
\end{verbatim}}\usageupspace

Integer to ASCII yields the base 10 representation of a number in a {\WOOL}
string.

{\exemplefont\begin{verbatim}
        (itoa 10)                       ==>     "10"
\end{verbatim}}

\ITEMc{key}{keypress}{keyrelease}{keyboard events}
        
{\usagefont\begin{verbatim}
(key keysym modifier)
(keypress keysym modifier)
(keyrelease keysym modifier)
\end{verbatim}}\usageupspace

Returns an event matching the press on a key of keysym code
\verb"keysym"\footnote{A keysym is a number representing the symbolic
meaning of a key. This can be a letter such as ``A'', or function keys, or
``Backspace'', ... The list of keysyms can be found in the {\bf keysymdef.h}
file in the {\tt /usr/include/X11} directory.}, with the
modifier\footnote{See the list of modifiers at the \verb"with-alt" entry,
page~\pageref{with-alt}} key {\tt modifier} depressed. The \verb"key" event
will wait for the corresponding keyrelease event before returning, while the
\verb"keypress" event will return immediately. Number or modifier can take
the value \verb"any", thus matching any of the values of the corresponding
argument.

Keysyms can be given as numbers or as symbolic names.

Examples:{\exemplefont\upspace\begin{verbatim}
        (key 0xff08 with-alt)   ; matches typing Alternate-Backspace
        (key "BackSpace" with-alt)       ; same effect
\end{verbatim}}

{\bf Note:} These functions convert the keysym to the appropriate keycode
for the keyboard, so you should do any re-mapping of keys via the
\seeref{set-key-binding} function {\bf before} using any of them.

\ITEMa{key-make}{makes a key symbol out of a descriptive name}

{\usagefont\begin{verbatim}
(key-make keyname)
\end{verbatim}}\usageupspace

Returns a {\bf keysym} (number) associated with the symbolic name
\verb"keyname" (string), to be used with the \verb"key", \verb"keypress",
\verb"keyrelease" and \verb"send-key-to-window" functions. The list of
symbolic names for keys can be found in the include file
``\verb"/usr/include/X11/keysymdef.h"''.

For instance, the backspace key is listed in \verb"keysymdef.h" as:
{\exemplefont\begin{verbatim}
        #define XK_BackSpace            0xFF08  /* back space, back char */
\end{verbatim}}
and you can specify the keysym for backspace with:
{\exemplefont\begin{verbatim}
        (key-make "BackSpace")          ==> 65288 (0xFF08)
\end{verbatim}}

\ITEMa{keycode-to-keysym}{converts a key code to its symbolic code}

{\usagefont\begin{verbatim}
(keycode-to-keysym code modifier)
\end{verbatim}}\usageupspace

This function returns the server-independent numeric code (the KeySym)
used to describe the key whose keyboard code is \verb"code" while the
modifier \verb"modifier" is down. 

{\bf Note:} Only know the \verb"alone" and \verb"with-shift" modifiers 
are meaningful in this {\bf X} function.

\ITEMa{keysym-to-keycode}{converts a symbolic code to a key code}

{\usagefont\begin{verbatim}
(keysym-to-keycode keysym)
\end{verbatim}}\usageupspace

This function returns the key code, i.e., the raw code sent by the displays
keyboard when the user depresses the key whose server-independent numeric
code is listed in the ``\verb"/usr/include/X11/keysymdef.h"'' include file
and given as a number as the \verb"keysym" argument.

\ITEMa{kill-window}{destroys a client}
        
{\usagefont\begin{verbatim}
(kill-window [window])
\end{verbatim}}\usageupspace

Calls XKillClient on the current window or \verb"window" if specified. This is
really a forceful way to kill a client, since all X resources opened by the
client owning the window are freed by the X server. If only you want to
remove a window of a client, use \seep{delete-window}.

\ITEMb{label-horizontal-margin}{label-vertical-margin}{margins around labels}
        
\usagetyped{Numeric variables}{number}

The value of these global variables are used by the \verb"label-make"
functions to add margins (given in pixels) around the string displayed.
\verb"label-horizontal-margin" defaults to 4 and
\verb"label-vertical-margin" to 2.

\ITEMa{label-make}{makes a pixmap by drawing a string}
        
{\usagefont\begin{verbatim}
(label-make label [font])
\end{verbatim}}\usageupspace

Creates a label with string \verb"label" drawn with \verb"foreground" color
in the font \verb"font" on a \verb"background" background. This function
builds a pixmap which is returned. 

Note that when used in a plug, the resulting pixmap is directly painted on
the background pixmap of the wob, speeding up the redisplay  since
re-exposure of the wob will be done directly by the server, but consuming a
small chunk of server memory to store the pixmap.

\contextdim{-2cm}{
foreground & color of the text string \\
background & color of the background \\
font & font of the string if not given \\
label-horizontal-margin &  \\
label-vertical-margin & margins around the string \\
}

\ITEMa{last-key}{last key pressed}
        
{\usagefont\begin{verbatim}
(last-key)
\end{verbatim}}\usageupspace

If the last event processed by a fsm was a key event, returns the string
generated by it. (Pressing on A yields the string \verb|"A"|). It uses 
{\bf XLookupString} to do the translation.

\ITEMa{length}{length of list or string}
        
{\usagefont\begin{verbatim}
(length list)
(length string)
\end{verbatim}}\usageupspace

Returns the number of elements of the list or the number of
characters of the string.

\ITEMa{list}{makes a list}
        
{\usagefont\begin{verbatim}
(list elt1 elt2 ... eltN)
\end{verbatim}}\usageupspace

Returns the list of its N evaluated arguments.

{\exemplefont\begin{verbatim}
        (list (+ 1 2) (+ "foo" "bar"))  ==>     (3 "foobar")
\end{verbatim}}

\ITEMa{list-make}{makes a list of a given size}

{\usagefont\begin{verbatim}
(list-make size elt1 elt2 ... eltN)
\end{verbatim}}\usageupspace

Returns the list of size \verb"size" (number) elements. The element of rank
\verb"i" is initialized to \verb"elt"$j$, where $j = i$ modulo $N$, if
elements are provided, \verb"()" otherwise.

{\exemplefont\begin{verbatim}
        (list-make 8 'a 'b 'c)          ==>     (a b c a b c a b)
        (list-make 3)                   ==>     (() () ())
\end{verbatim}}

\ITEMa{list-of-screens}{list of managed screens}

{\usagefont\begin{verbatim}
(list-of-screens)
\end{verbatim}}\usageupspace

Returns the list of screens actually managed by {\GWM}.

\ITEMa{list-of-windows}{returns the list of managed windows}
        
{\usagefont\begin{verbatim}
(list-of-windows ['window] ['icon] ['mapped]
    ['stacking-order])
\end{verbatim}}\usageupspace

Returns the list of all windows and icons managed by {\GWM}, mapped or not.
Called without arguments, this function returns the list of all windows (not
icons), mapped or not. It can take the following atoms as optional
arguments:

\begin{description}
\item[window] lists only main windows
\item[icon] lists only realized icons
\item[mapped] lists only currently mapped (visible) windows or icons
\item[stacking-order] lists windows in stacking order, from bottommost
(first) to topmost (last). This option is slower than the default which is
to list them by the order in which they were managed by {\GWM}, from the
the newest to the oldest.
\end{description}

The \verb"window" and \verb"icon" arguments are mutually exclusive.

{\bf Note:} The two following expressions do not return the same list:
{\exemplefont\begin{verbatim}
        (list-of-windows 'icon)
        (mapfor w (list-of-windows) window-icon)
\end{verbatim}}
The first one will return the list of all realized icons, that is only the
icons of windows that have already been iconified at least once, but the
second one will trigger the realization of the icons of {\bf all} the
managed windows, by the evaluation of \verb"window-icon", on all the
windows.

{\bf Note:} The main window of an application is always realized, even if
the application started as iconic.

\ITEMa{load}{loads and executes a {\WOOL} file}
        
{\usagefont\begin{verbatim}
(load filename)
\end{verbatim}}\usageupspace

Loads and executes the {\WOOL} file given in the filename string argument,
searching through the  path specified by the {\bf GWMPATH} variable or the
{\bf -p} command line switch. Defaults to \verb".:$HOME:$HOME/gwm:INSTDIR",
where \verb"INSTDIR" is your local {\GWM} library directory, which is normally
\verb"/usr/local/X11/gwm", but can be changed by your local installer.

On startup, {\GWM} does a \verb|(load ".gwmrc")|.

Returns the complete pathname of the file as a string if it was found, ()
otherwise but does not generate an error, only a warning message.

Searches first for \verb"filename.gwm" then \verb"filename" in each
directory in the path. If the filename includes a \verb"/" character, the
file is not searched through the path. If any error occurs while the file is
being read, {\WOOL} displays the error message and aborts the reading of
this file. This implies that you can't expect {\GWM} to run normally if
there has been an error in the .gwmrc. (You can turn off this behavior and
make {\GWM} continue reading a file after an error at your own risk, with
the {\bf -D} command line option).

\ITEMa{lower-window}{lowers window below another window}
        
{\usagefont\begin{verbatim}
(lower-window [window])
\end{verbatim}}\usageupspace

Lowers  the current window below all other top-level windows or, if the
\verb"window" argument is present, just below the window given as argument.


\ITEMa{map-window}{maps window}
        
{\usagefont\begin{verbatim}
(map-window [window])
\end{verbatim}}\usageupspace

Maps (makes visible) the window (or current window). Does not raise it.

\ITEMa{match}{general regular expression matching package}
        
{\usagefont\begin{verbatim}
(match regular-expression string [number] ...)
\end{verbatim}}\usageupspace

This is the general function to match and extract substrings out of a {\WOOL}
string. This string is matched against the pattern in regular expression,
and returns () if the pattern could not be found in the string, and the
string otherwise.

If \verb"number" is given, match returns the sub-string 
matching the part of the
regular-expression enclosed in between the number-th open parenthesis and
the matching closing  parenthesis (first parenthesis is numbered 1). Do not
forget to escape twice the parentheses, once for {\GWM} parsing of strings, and
once for the regular expression, e.g., to extract ``bar'' from ``foo:bar'' you
must use: \verb|(match ":\\(.*\\)" "foo:bar" 1)|

If more than one \verb"number" argument is given, returns a list made of
of all the specified sub-strings in the same way as for a single \verb"number"
argument. For instance, to parse a X geometry such as \verb"80x24+100+150"
into a list \verb"(x y width height)", you can define the following function:

{\exemplefont\begin{verbatim}
        (defun parse-x-geometry (string)
          (mapfor dim                   
            (match "=*\\([0-9]*\\)x\\([0-9]*\\)\\([-+][0-9]*\\)\\([-+][0-9]*\\)" 
                   string 3 4 1 2)
            (atoi dim))
        
        (parse-x-geometry "80x24+100+150")  ==>  (100 150 80 24)
\end{verbatim}}

Note that this function will return \verb"nil" if there is a syntax error in
the given string.

The accepted regular-expressions are the same as for ED or GREP, viz:

\begin{enumerate}

\item  Any character except a special character matches itself.  Special
characters are the regular expression delimiter plus \verb|\ [ .| and
sometimes \verb|^ * $|.

\item  A \verb"." matches any character.

\item  A \verb|\| followed by any character except a digit or \verb|( )|
matches that character.

\item  A nonempty string \verb|s| bracketed \verb|[s]| (or \verb|[^s]|)
matches any character in (or not in) \verb|s|. In \verb|s|, \verb|\| has no
special meaning, and \verb|]| may only appear as the first letter. A
substring \verb|a-b|, with \verb|a| and \verb|b| in ascending ASCII order,
stands for the inclusive range of ASCII characters.

\item  A regular expression of form {\bf 1-4} followed by \verb|*| matches a
sequence of 0 or more matches of the regular expression.

\item  A regular expression, \verb"x", of form {\bf 1-8}, bracketed
\verb"\(x\)" matches what \verb"x" matches.

\item  A \verb"\" followed by a digit \verb|n| matches a copy of the string
that the bracketed regular expression beginning with the \verb|n|th
\verb|\(| matched.

\item  A regular expression of form {\bf 1-8}, \verb"x", followed by a
regular expression of form \verb"1-7", \verb"y" matches a match for \verb"x"
followed by a match for \verb"y", with the \verb"x" match being as long as
possible while still permitting a \verb"y" match.

\item  A regular expression of form {\bf 1-8} preceded by \verb"^" (or
followed by \verb"$"), is constrained to matches that begin at the left (or
end at the right) end of a line.

\item  A regular expression of form {\bf 1-9} picks out the longest among
the leftmost matches in a line.

\item  An empty regular expression stands for a copy of the last regular
expression encountered.

\end{enumerate}

\ITEMa{member}{position of element in list or in string}

{\usagefont\begin{verbatim}
(member element list)
(member substring string)
\end{verbatim}}\usageupspace

In the first form, scans the list (with the \verb"equal" predicate) to find
the object \verb"element". If found, returns its index in the list (starting
at 0), \verb"()" otherwise.

In the second form, looks for the first occurrence of the string
\verb"substring" in the string \verb"string", and returns the character
position at which it starts, or \verb"()" if not found.

\ITEMa{meminfo}{prints memory used}
        
{\usagefont\begin{verbatim}
(meminfo)
\end{verbatim}}\usageupspace

Prints the state of the malloc allocator of all dynamic memory used, by
{\WOOL} or by the Xlib. You will note that reloading your profile consumes
memory.  This is unavoidable.

\ITEMa{menu}{menu associated with wob}
        
\usagetyped{Variable}{menu}

The value of this global variable is used by the constructors of all wobs to
set the associated menu of the wob.  It should be set to a menu constructed
with \verb"menu-make" or ().  This menu is the default one used by the
\verb"pop-menu" and \verb"unpop-menu" functions.

\ITEMa{menu-make}{makes a menu}
        
{\usagefont\begin{verbatim}
(menu-make bar1 bar2 ... barN)
\end{verbatim}}\usageupspace

Creates a menu, vertical or horizontal depending on the value of the context
variable \verb"direction" ({\tt horizontal} or {\tt vertical}). This
function returns a menu descriptor and at the same time realize the
corresponding menu wob (to get the
wob, use the \verb"menu-wob" function below). Unlike the plug-make or
bar-make functions that return the {\WOOL} description to be used to create
different wob instances, this is the only {\WOOL} function which really
creates a wob (as an unmapped X window) on execution.

A vertical menu is composed of horizontal bars stacked on top of each other.
A horizontal menu is composed of vertical bars aligned from left to right.
The menu takes the width (or height) of the largest bar, then adjusts the
others accordingly. 

\context{
fsm & its finite state machine \\
direction & for the direction of the menu \\
borderwidth & the width of the menu's border in pixels \\
borderpixel & its color \\
bordertile & if it has a pixmap as border pattern \\
background & the color of the background \\
bar-separator & the number of pixels between consecutive bars \\
tile &  the pattern of its background (may be ()) \\
cursor & the cursor's shape when in it \\
property & the property associated with the menu \\
}

\ITEMa{menu-wob}{returns wob associated with menu}

{\usagefont\begin{verbatim}
(menu-wob menu-descriptor)
\end{verbatim}}\usageupspace

Returns the wob representing the menu as given by the descriptor returned by
\verb"menu-make" and used in the context variable \verb"menu".

\ITEMa{meter}{sets meter attributes}

{\usagefont\begin{verbatim}
(meter [key value]...)
\end{verbatim}}\usageupspace

Sets the attributes of the current screen meter. The key must be an atom, and
the corresponding attribute is set to the value.

\desc{horizontal-margin}{Key}{Value}{
 font & the font of the text displayed in the meter\\
 background & the background color of the meter\\
 foreground & the color the text will be drawn with\\
 horizontal-margin & in pixels between text and sides of meter\\
 vertical-margin & in pixels between text and top and bottom of meter\\
}

{\exemplefont\begin{verbatim}
        (meter 'font (font-make "9x15")
                'foreground (color-make "blue"))
\end{verbatim}}

\ITEMa{meter-close}{unmaps the meter}
        
{\usagefont\begin{verbatim}
(meter-close)
\end{verbatim}}\usageupspace

Closes the meter (makes it disappear).

\ITEMa{meter-open}{displays the meter}
        
{\usagefont\begin{verbatim}
(meter-open x y string)
\end{verbatim}}\usageupspace

Displays the meter at location x,y. The string argument is only used to set
the minimum width of the meter, call meter-update to display a string in it.

\ITEMa{meter-update}{writes a string in the meter}
        
{\usagefont\begin{verbatim}
(meter-update string)
\end{verbatim}}\usageupspace

Change the string displayed in the meter to \verb"string". Updates the width
of the meter accordingly. Since the meter is generally used in a context
where speed is important, it is never shrunk, only expanded.

\ITEMb{move-grid-style}{resize-grid-style}{style of grid for move and resize}
        
\usagetyped{Numeric variables}{number}

These variables control the kind of grid which is displayed on move or
resize operations. Currently available grids are:

\desctabledim{1cm}{Value}{Style of grid}{
0 &     outer rectangle of the window (default) \\
1 &     outer rectangle divided in 9 (uwm-style) \\
2 &     outer rectangle with center cross (X) inside \\
3 &     outer rectangle + inner (client) outline \\
4 &     styles 1 and 3 combined \\
5 &     2 pixel wide outer rectangle\\
}

\ITEMb{move-meter}{resize-meter}{shows meter for move and resize}

\usagetyped{Numeric variables}{number}

If set to 1, the meter is shown during an interactive move,
displaying the coordinates of the moved window.

\ITEMa{move-window}{moves window interactively or not}
        
{\usagefont\begin{verbatim}
(move-window)
(move-window window)
(move-window x y)
(move-window window x y)
\end{verbatim}}\usageupspace

Moves the window. If coordinates are specified, moves directly the current
window (or \verb"window" argument if specified) to \verb"x,y" (upper-left
corner coordinates in pixels). If coordinates are not specified, moves the
window interactively, i.e., display the grid specified with the current
value of \verb"move-grid-style", and track the grid, waiting for a
buttonrelease for confirmation or a buttonpress for abort. If confirmed, the
window is moved to the latest grid position. The cursor will take the shape
defined by the variable \verb"cursor" if not ().  When move-window is called
on another event than buttonpress, there is no way to abort the move.

If you have a strange behavior on move or resize window, check if you
didn't trigger them with a \verb"button" event instead of a
\verb"buttonpress", since {\GWM} will wait for a release already eaten by
move-window!

When used interactively (first two forms), the return value can be one of:

\desctabledim{1cm}{Return}{Case}{
()&     Ok\\
0 &     X problem (window disappeared suddenly, other client 
        grabbing display, etc.)\\
1 &     pointer was not in the window screen\\
2 &     user aborted by pressing another button\\
3 &     user released button before the function started\\
}

\ITEMa{name-change}{event generated when window changes its name}
        
\usagetyped{Constant}{event}

This event is sent to the window when the client changes its name.  The
window should then warn the appropriate wobs, by a \see{send-user-event} if
needed. It is equivalent to the event made by the call 
\verb|(property-change "WM_NAME")|.

\ITEMa{namespace-make}{creates a namespace}

{\usagefont\begin{verbatim}
(namespace-make)
\end{verbatim}}\usageupspace

Creates a new {\bf namespace} and returns it. A namespace is a set of
variable names, called {\bf names}, that have different values for each
state in which the namespace can be. 

The special namespace \verb"screen." has as many spaces as there are
screens, and its current state is always updated to match the current screen
by {\GWM}. Other namespaces must switch states by the \verb"namespace"
function.

For instance, to define a namespace \verb"application." with a name
\verb"application.border" which will depend on the current application:

{\exemplefont\begin{verbatim}
        (setq application. (namespace-make))
        (setq application.clock (namespace-add application.))
        (setq application.load (namespace-add application.))
        (defname 'application.border application.)
        (namespace application. application.clock)
        (setq application.border 1)
        (namespace application. application.load)
        (setq application.border 2)

        (namespace application. application.clock)
        application.border                                ==>  1
        (namespace application. application.load)
        application.border                                ==>  2
\end{verbatim}}

\ITEMa{namespace-add}{adds a state to a namespace}

{\usagefont\begin{verbatim}
(namespace-add namespace)
\end{verbatim}}\usageupspace

Adds a new state to the argument namespace and returns the index (numeric
offset starting at 0) of the newly created state. This index should then be
used to set the current state of the namespace by the \verb"namespace"
function.

\ITEMa{namespace-remove}{removes a state from a namespace}

{\usagefont\begin{verbatim}
(namespace-remove namespace state-index)
\end{verbatim}}\usageupspace

Destroys a state (made with \verb"namespace-add" of a namespace. The number
\verb"state-index" is the index returned by namespace-add at the creation of
the state.

\ITEMa{defname}{declares a name in a namespace}

{\usagefont\begin{verbatim}
(defname name namespace [value])
\end{verbatim}}\usageupspace

Defines the atom \verb"name" to be a name in the namespace. If the value is
given, for each state of the namestate, a \verb"(set name (eval value))" is
done, otherwise the value of name is left undefined in all the states.

Suppose you want to have a variable \verb"dpi" giving the density of the
screen in dots per inches for each screen:

{\exemplefont\begin{verbatim}
        (defname 'dpi screen.
            (/ (* screen-width 254) 
                (* screen-widthMM 10)))
\end{verbatim}}

\ITEMa{namespace}{sets current state of a namespace}

{\usagefont\begin{verbatim}
(namespace namespace current-state-index)
\end{verbatim}}\usageupspace

Sets the current state of the namespace argument to the index returned
from the
\verb"namespace-add" function at the creation of the state. If the index is
out of bounds (like ``-1''), returns the current index.

\ITEMa{namespace-of}{returns namespace of the name}

{\usagefont\begin{verbatim}
(namespace-of name)
\end{verbatim}}\usageupspace

Returns the namespace where the name \verb"name" is declared in, otherwise
(if it is a plain atom or an active value) returns \verb"()".

\ITEMa{namespace-size}{number of states in the namespace}

{\usagefont\begin{verbatim}
(namespace-size namespace)
\end{verbatim}}\usageupspace

Returns the number of the states of the namespace, i.e., the number of
possible values for each name in this namespace. The legal values for the
index to be given to the \verb"namespace" function are in the range
from 0 to the return value - 1.

\ITEMa{not}{logical negation}
        
{\usagefont\begin{verbatim}
(not object)
\end{verbatim}}\usageupspace

Logical negation. Returns \verb"t" if \verb"object" is \verb"()",
\verb"()" otherwise.

\ITEMa{oblist}{prints all defined objects}
        
{\usagefont\begin{verbatim}
(oblist)
\end{verbatim}}\usageupspace

Prints the name and value of all currently defined {\WOOL} atoms.

\ITEMb{on}{on-eval}{triggers a transition on an event in a state of a fsm}
        
{\usagefont\begin{verbatim}
(on event [action [state]])
(on-eval event [action [state]])
\end{verbatim}}\usageupspace

This is the function used to build transitions in the states of the fsms.
The \verb"on" evaluates only its \verb"event" argument, so that it is not
necessary to quote the \verb"action" or \verb"state" arguments, whereas
\verb"on-eval" evaluates all its arguments.

When an event is handled by an fsm, it checks sequentially all the
transitions of the current state for one whose \verb"event" field matches
the incoming event. If found, it calls the {\WOOL} function found in the
\verb"action" field, which must be a function call, and makes the
state argument the current state of the fsm. If no  state is given, it is
taken as the same state of the fsm. If no action is given, no action is
performed.

The destination state is an atom which should have a state value defined in
the current fsm (i.e., set by setq of an atom 
with the result of a \verb"state-make") at
the time when the fsm is evaluated, which means that you can make forward
references to states, as in the following example.

Example: To define a behavior of a wob cycling through 3 states on the
click of any button, use this fsm:

{\exemplefont\begin{verbatim}
        (fsm-make 
                (setq state1 (state-make
                        (on (button any any) (? 1) state2))
                (setq state2 (state-make
                        (on (button any any) (? 2) state3)))
                (setq state3 (state-make
                        (on (button any any) (? "3.\n") state1))))
\end{verbatim}}

\ITEMb{opening}{closing}{{\WOOL} hooks on creation and deletion of windows}

\usagetyped{Variables}{Wool}

The values of these global variables are used by the \see{window-make}
constructor to define {\WOOL} expressions which are evaluated just before
creating or just after destroying a window or an icon. 

{\bf Note:} The \verb"opening" field of a window is always evaluated on the
creation of this window, thus the \verb"opening" of an icon is only
executed on the first iconification of the icon.

\ITEMa{or}{logical OR of expressions}
        
{\usagefont\begin{verbatim}
(or object1 object2 ... objectN)
\end{verbatim}}\usageupspace

Logical OR. Returns the first non-() object or ().

\ITEMa{pixmap-load}{loads an X pixmap from an ascii file}
        
{\usagefont\begin{verbatim}
(pixmap-load filename)
\end{verbatim}}\usageupspace

This function builds a pixmap by reading its description in the file
\verb"filename.xpm" or \verb"filename", searched in the directory path held
by the {\bf GWMPATH} shell variable \seesnp{GWMPATH}. 
In case of
error the default pixmap returned is the same as the one returned by the
\verb"pixmap-make" function.

The pixmap is expected to be described in the {\bf XPM} (for X PixMap)
format. For instance such a file for a \verb"foo" pixmap, 6x3 with four
colors would look like:

{\exemplefont\begin{verbatim}
        #define foo_paxformat 1
        #define foo_width 6
        #define foo_height 3
        #define foo_ncolors 4
        #define foo_chars_per_pixel 2
        static  char * foo_colors[] = {
        "  " , "DarkSlateGrey",
        ". " , "#A8A8A8",
        "XX" , "White",
        "o " , "#540054005400"  
        } ;
        static char * foo_pixels[] = {
        "  XXXX.     ",
        "XX  o . XXXX",
        "o o o XXXX. "
        } ;
\end{verbatim}}

{\bf Note:} In the current implementation, only \verb"paxformat" 1 and
\verb"chars_per_pixel" 2 are supported.

{\bf Note:} The definition of such a format is freely available from us,
with the appropriate C routines to read/write it.

\ITEMa{pixmap-make}{builds a pixmap (color image)}
        
{\usagefont\begin{verbatim}
(pixmap-make filename)
(pixmap-make width height)
(pixmap-make background-color file1 color1 
             file2 color2 ... )
\end{verbatim}}\usageupspace

With a single argument, loads the bitmap file given in the filename string
argument, searching through the bitmap path specified by the {\bf GWMPATH}
variable or the {\bf -p} command line switch \seesnp{GWMPATH}. 
Searches first for
\verb"filename.xbm" then \verb"filename" in each directory in the path. If
the filename includes a \verb"/" character, the file is not searched through
the path.
The pixmap is built by drawing the ``unset'' bits of the bitmap with the 
color held by the \verb"background" variable and  the ``set'' bits with
the color held by the \verb"foreground" variable.

When used with two arguments, \verb"pixmap-make" returns a newly
created pixmap of width \verb"width" and height \verb"height" filled with
the current \verb"foreground" color.

When used with three or more arguments,
returns the pixmap constructed by using the bitmaps given as arguments to
paint only the ``set'' bits of bitmap in file file{\it i}
(searched along the same rules as for the first form) to the color{\it
i}.  The ``unset'' bits remain untouched. The pixmap is first painted with the
color given in the \verb"background-color" argument.

If the specified bitmap cannot be loaded, either because the file cannot be
found or does not contain a bitmap, a default built-in bitmap is used and a
warning is issued. The default is the MIT X Consortium logo as a 20x15 bitmap.
Bitmaps can be of any size, they will be centered in the resulting pixmap
which will be of the maximum size of its components.

The filename arguments can also be replaced by:

\begin{description} 

\item[active-labels] The string is centered on the bitmap, drawn in the
following color.

\item[pixmaps]      The pixmap is painted in the center, with its own
background color (note that \verb"label-make" returns a pixmap!). The
following color is then ignored, but {\bf must} be specified for consistency
purposes.

\end{description}


\ITEMa{place-menu}{maps a menu as a normal client window}
        
{\usagefont\begin{verbatim}
(place-menu name menu [x y])
\end{verbatim}}\usageupspace

Places a menu (made with \verb"menu-make") on the screen. This window is
then managed like any other client window. Its client class is \verb"Gwm",
its client name \verb"menu", and window name given by the parameter 
\verb"name".  It is
placed at position {\tt x, y} if specified and at 0,0 otherwise
(NB: It is placed at last popped
position if it was a popped menu). Returns the created wob.

{\bf WARNING:} Be careful not to mix popup menus and placed ones, calling
\verb"pop-menu" on an already ``placed-menu'' will result in unpredictable
behavior. 
\context{
\label{icon-name}   icon-name & the name of the icon for the window. 
If (), uses {\tt name}.\\
\label{starts-iconic}   starts-iconic & if the menu will first appear as 
an icon\\
}

\ITEMa{plug-make}{makes a plug}
        
{\usagefont\begin{verbatim}
(plug-make pixmap)
(plug-make active-label)
\end{verbatim}}\usageupspace

This function builds a plug, the atomic wob object. It is composed of a
graphic object (either a pixmap, created with pixmap-make or label-make, or
another type of object such as an ``active-label''). The size of the
graphic object determines the size of the plug.  Thus if you change the
pixmap of the plug via \verb"wob-tile", the plug is resized
accordingly.

\context{
            fsm       &     its finite state machine \\
            borderwidth  &  the width of the plug's border in pixels \\
            borderpixel  &  its color \\
            bordertile   &  the pixmap to paint the border with \\
            background   &  the color of the background \\
            menu         &  the pop-up associated with it  \\
            cursor       &      the cursor's shape when in it \\
            property     &  the property associated with the plug \\
}

\ITEMa{plug-separator}{inter-plug space in bars}
        
\usagetyped{Numeric variable}{number}

The value of this global variable is used by the \verb"bar-make" function to
give the space in pixels between 2 consecutive plugs not separated by
extensible space.

\ITEMa{pop-menu}{pops a menu}
        
{\usagefont\begin{verbatim}
(pop-menu [menu] [position] ['fixed])
\end{verbatim}}\usageupspace

This function grabs the server (using the \see{grab-server} function), thus
preventing other client requests to be processed by the server, and maps the
given menu (or the menu associated with the current wob if menu
is not given or is set to ()). The menu is guaranteed to be on the screen
when this function returns.
The behavior of the menu is then determined by
the menu's fsm. The cursor will take the shape defined by the variable
\verb"cursor" if it is  non-nil.

The \verb"position" number is the item in which you want the cursor to appear
(centered), the first item is numbered 0.

If the atom \verb"fixed" is given, the menu is not moved under the current
position of the pointer, it stays at the last position it was. So to pop a
menu \verb"menu" at (67, 48), do:

{\exemplefont\begin{verbatim}
        (move-window (menu-wob menu) 67 48)
        (pop-menu menu 'fixed)
\end{verbatim}}

The current wob at the time of the call to \verb"pop-menu" becomes the
parent of the menu.

{\bf Note:} A menu is not a {\GWM} window, so when a menu is popped, 
the current
window becomes the window parent of the wob which has popped the menu.

{\bf Note:} The arguments to \verb"pop-menu" can be given in any order and
are all optional.

\ITEMa{print-errors-flag}{controls printing of error messages}

\usagetyped{Numeric variable}{number}

If nonzero, {\WOOL} error messages are printed on {\GWM} output device,
which is the default. \verb"error-occurred" sets this variable to 0 to
prevent printing errors.

\ITEMa{print-level}{controls printing depth of lists}

\usagetyped{Numeric variable}{number}

This variable controls the maximum depth at which the lists will be printed.

{\exemplefont\begin{verbatim}
        (setq print-level 2)
        '(1 (2 (3 (4 rest))))           ==>      (1 (2 (...)))
\end{verbatim}}


\ITEMa{process-events}{recursively process all pending events\hfill{\bf EXPERT}}

{\usagefont\begin{verbatim}
(process-events)
\end{verbatim}}\usageupspace

This function reads the event queue and recursivly processes all pending
events by re-entering the main {\GWM} loop. It returns when there are no
more pending events on the X event queue.

This functions allows a pseudo-multitasking capability for {\GWM}.
You can thus implement ``background jobs'' such as a general desktop space
cleaning routine, to be called after each move, opening, or closing function
that pauses by calling \verb"process-events" at regular intervals to let
{\GWM} do its other tasks. You should then take into account the fact that
the current function could be recursively called by another move operation
triggered by an event processed by \verb"process-events", however and thus
be re-entrant.

\ITEMa{process-exposes}{treats all pending expose events}
        
{\usagefont\begin{verbatim}
(process-exposes)
\end{verbatim}}\usageupspace

This function reads the event queue and processes all expose events it finds
for {\GWM} objects on the screen. It is used by pop-menu to be sure that the
menu is completely drawn before letting it react to user events.

\ITEMa{rotate-cut-buffers}{rotate server cut buffers}

{\usagefont\begin{verbatim}
(rotate-cut-buffers number)
\end{verbatim}}\usageupspace

Exchange the contents of the {\bf 8} cut buffers on the X server so that
buffer $n$ becomes buffer $n + $ \verb"number" modulo 8.

\ITEMa{send-current-event}{re-sends X event to the client of a window}

{\usagefont\begin{verbatim}
(send-current-event window)
\end{verbatim}}\usageupspace

Re-sends the last key or button event to the client of the \verb"window"
argument. If \verb"window" is \verb"()", it is taken as the current window.

\ITEMbi{\{\}}{progn}{sequence of instructions}
        
{\usagefont\begin{verbatim}
(progn inst1 inst2 ... instN)
{inst1 inst2 ... instN}
\end{verbatim}}\usageupspace

The classical Lisp \verb"progn" function, evaluating all its arguments and
returning the result of the last evaluation. Useful in places where
more then one Lisp instruction is expected, for instance in
\verb"then" fields of the \verb"if" instruction, or in the
\verb"action" field of the \verb"on" function. The brace notation is
just a shortcut for writing \verb"progn" sequences.

\ITEMa{property-change}{event generated when a client window changes a property}
        
{\usagefont\begin{verbatim}
(property-change property-name)
\end{verbatim}}\usageupspace

This event is sent when the property of name \verb"property-name" is changed
on the client window.

\ITEMa{raise-window}{raises window on top of other windows}
        
{\usagefont\begin{verbatim}
(raise-window [window])
\end{verbatim}}\usageupspace

Raises  the current window on top of all other top-level windows or, if the
\verb"window" argument if present, above the window given as argument.

\ITEMa{re-decorate-window}{re-decorates the client window by {\GWM}}
        
{\usagefont\begin{verbatim}
(re-decorate-window [window])
\end{verbatim}}\usageupspace

Un-decorate the client window and re-decorate it as if it had appeared on
the screen. Useful after a \verb|(load ".gwmrc")| to test quickly any
modifications to your profile.

\ITEMa{refresh}{refreshes the screen}
        
{\usagefont\begin{verbatim}
(refresh [window])
\end{verbatim}}\usageupspace

If a window argument is provided, refreshes only this window, otherwise 
forces a redraw of the screen like XREFRESH(1). Refresing is done by mapping
and unmapping a new window over the aera to refresh.

\ITEMa{replayable-event}{makes a replayable event from a normal event}

{\usagefont\begin{verbatim}
(replayable-event event)
\end{verbatim}}\usageupspace

Makes the given button or key event replayable
\seesnp{ungrab-server-and-replay-event}. This is not the default for events
because replayable events freeze the server state for {\GWM} when a passive
grab is activated with them, which might not be desirable.

{\exemplefont\begin{verbatim}
        (set-grabs (replayable-event (button 1 alone)))
\end{verbatim}}

\ITEMa{resize-style}{style of interactive resize}

\usagetyped{Numeric variable}{number}

This variable controls the way the interaction with the user is handled
during resizes, and can take the numeric values:

\desctabledim{1cm}{Value}{Style of Resize}{
0 & is a {\tt Uwm}-like resize, i.e. the window is divided
in nine regions, and you are allowed to drag which side or corner of the
window you were in when the resize began.\\
1 & \sloppy is a {\bf Mwm}-like resize, i.e. ou resize the window by its
side or corner, the width of the corners being determined (in pixels) by the
value of the global variable 
\verb"mwm-resize-style-corner-size"\label{mwm-resize-style-corner-size}. 
If you are dragging the side of the window and you go at less than
\verb"mwm-resize-style-corner-size" of a corner, you then drag this corner,
if the global variable
\verb"mwm-resize-style-catch-corners"
\label{mwm-resize-style-catch-corners} is nonzero.

The corner or side dragged is reflected by the shape of the cursor you have
put in the global variables cursor-NW, cursor-NE, cursor-N, cursor-SW,
cursor-SE, cursor-S, cursor-E, cursor-W (see p~\pageref{cursor-NW}). While
you are not dragging anything, the cursor is still determined by the value
of {\tt cursor}
\\
}

\ITEMa{resize-window}{resizes the window interactively or not}
        
{\usagefont\begin{verbatim}
(resize-window)
(resize-window window)
(resize-window width height)
(resize-window window width height)
\end{verbatim}}\usageupspace

Resizes the window. If the dimensions are specified, resizes directly the
current window (or the given \verb"window" argument) to the given size,
specified in pixels, rounded down to the allowed size increments (if you
want to resize by increments, use the \verb"window-size" active value). If
the dimensions are not specified, resizes the window interactively, i.e.,
display the grid specified with the current value of \verb"resize-grid-style",
and track the grid, waiting for a buttonrelease for confirmation or a
buttonpress for abort. If confirmed, the window is resized to the latest
grid position. The cursor will take the shape defined by the variable
\verb"cursor", if it is non-nil.

{\bf NOTE:} The dimensions given to resize-window are those of the outer
window, in pixels, including the {\GWM} decoration. Use the 
\verb"window-size" active value if you want to specify the client dimensions.

{\bf WARNING:} When resize-window is called on an event other than
buttonpress, there is no way to abort the resize.

The interactive resize interaction is set by the value of the global variable
\verb"resize-style"

When used interactively (first two forms), the return value can be one of:

\desctabledim{1cm}{Returns}{Case}{
()&     Ok\\
0 &     X problem (window disappeared suddenly, other client
        grabbing display, etc.)\\
1 &     pointer was not in the window screen\\
2 &     user aborted by pressing another button\\
3 &     user released button before the function started\\
}

\ITEMa{resource-get}{searches {\GWM} database for a resource}
        
{\usagefont\begin{verbatim}
(resource-get name class)
\end{verbatim}}\usageupspace

Calls the X11 resource manager for an object of name \verb"name" and
class \verb"class" (strings of dot-separated names) previously stored
by a call to \verb"resource-put" matching the description (See the
X11 documentation for precise description of the rule-matching
algorithm of the resource manager).

This is the recommended way to retrieve the window descriptions associated
with a client in the \verb"describe-window" function.

\ITEMa{resource-put}{puts a resource in {\GWM} database}
        
{\usagefont\begin{verbatim}
(resource-put name value)
\end{verbatim}}\usageupspace

Puts the value in the {\GWM} resource database so that it can be retrieved by a
future call to resource-get. Names can use the same conventions as in the
.Xdefaults file for normal X11 clients.

This is the recommended way to store the window descriptions associated with a
client so that they can be retrieved later in the \verb"describe-window"
function.

\ITEMa{restart}{restarts {\GWM}}
        
{\usagefont\begin{verbatim}
(restart)
(restart "prog" "arg1" "arg2" ... "argn")
\end{verbatim}}\usageupspace

Without arguments, restarts {\GWM}, more precisely, it does an EXEC(2) of
{\GWM} with the same arguments that were given on startup.

With arguments, terminates {\GWM} and starts a new process with the given
command-line arguments. This useful to restart {\GWM} with another profile,
as in:

{\exemplefont\begin{verbatim}
        (restart "gwm" "-f" "another-profile")
\end{verbatim}}

\ITEMa{root-window}{the root window}
        
\usagetyped{Active value}{window id}

Holds the wob describing the root window of the current screen. 
For instance, to set the background
pixmap of the screen to the bitmap in file ``ingrid'', say:

{\exemplefont\begin{verbatim}
        (with (wob root-window) (setq wob-tile (pixmap-make "ingrid")))
\end{verbatim}}

When set to a root window wob, sets the current wob and window to it,
and screen to the screen it belongs to.
 
\ITEMa{save-yourself}{asks client to update its WM\_COMMAND property}

{\usagefont\begin{verbatim}
(save-yourself [window])
\end{verbatim}}\usageupspace

Sends to the current window's (or to given window's) client window 
the ICCC message \verb"WM_SAVE_YOURSELF", telling the
application to update its \verb"WM_COMMAND" X property to a command line
which should restart it in its current state.

This function returns \verb"t" if the window supported this protocol, and
hence is supposed to update its \verb"WM_COMMAND" property; \verb"()"
otherwise \seesp{get-wm-command}.

\ITEMa{screen}{current screen}

\usagetyped{Active value}{screen number}

Returns or sets the current screen as the screen number (the same number as
\verb"X" in \verb"DISPLAY=unix:0.X"). Setting the screen also sets the
active values \verb"wob" and \verb"window" to the root window of the screen.
See the warning about using {\tt screen} in {\tt with} constructs under the
{\tt window} entry, p~\pageref{window}.

\ITEMa{screen-count}{number of screens attached to the display}

\usagetyped{Constant}{number}

\sloppy The number of screens attached to the display. Not equal to
\verb"(length (list-of-screens))" if you excluded screens by the \verb"-x"
command line option.

\ITEMc{screen-depth}{screen-height}{screen-width}{screen dimensions}
        
\usagetyped{Constants}{number}

These numerical variables hold the size of the current screen in bitplanes
(\verb"screen-depth") and pixels (\verb"screen-width" and
\verb"screen-height"). A screen-depth of 1 means that you are on a
monochrome screen.

\ITEMb{screen-heightMM}{screen-widthMM}{actual screen size in millimeters}

\usagetyped{Constants}{number}

Hold the dimensions of the screen in millimeters.

\ITEMa{screen-type}{visual type of screen}

\usagetyped{Active value}{atom --- not settable}

Returns the screen visual type as an atom, which can be either \verb"color",
\verb"gray", or \verb"mono", if the screen is a color, gray scale, or
monochrome device.

\ITEMb{send-key-to-window}{send-keycode-to-window}{sends key event to a client}
        
{\usagefont\begin{verbatim}
(send-key-to-window keysym modifier)
(send-key-to-window string modifier)
(send-keycode-to-window keycode modifier)
\end{verbatim}}\usageupspace

These three functions are used to send key events to the client of the
current window, to define function keys in {\GWM}, or to implement mouse
positioning of the cursor in Emacs for instance. The third form sends the
keycode (as returned by current-event-code), the first the keysym as defined in
the include file ``keysymdef.h''. Both keysyms and keycodes must be {\bf
numbers}, not names. You can however specify the symbolic name of the key
with the \verb"key-make" function. The \verb"modifier" parameter indicates 
which
modifier keys or which buttons were supposed to be down for the event.

The second form sends each character of the string (this works for
ASCII characters only) with the modifier \verb"modifier" to the window.

\ITEMa{send-user-event}{sends a {\GWM} ``user'' event}
        
{\usagefont\begin{verbatim}
(send-user-event atom [wob [do-not-propagate?]])
\end{verbatim}}\usageupspace

This function sends a ``user'' event, to the current window, if no argument
is present, or to the wob specified as argument. A user event is a {\GWM}
concept and is {\bf NOT} an X event, i.e., it is not seen by the server and is
immediately processed before \verb"send-user-event" terminates. The
peculiarity of an user-event is that it recursively propagates downwards in
the wob tree from the destination wob, the event being sent first to the
child, then to the wob itself. That is, if you send an user-event to the window, all
its decorations will receive it. The atom, which is evaluated by the
function, and thus needs to be quoted, is used merely as a kind of label for
the event.

If you provide a non-nil third argument, the event is sent to the given wob,
but is not propagated to its sons.

Example: 
{\exemplefont\upspace\begin{verbatim}
        (send-user-event 'get-focus)
\end{verbatim}}

{\bf Note:} \verb"send-user-event" saves and restores the current wob,
window, screen, and event. Thus if a piece of code triggered in another fsm sets the
current wob to another one, it will be restored to its previous value on
exit of the calling \verb"send-user-event".

\ITEMa{set-acceleration}{sets mouse speed}
        
{\usagefont\begin{verbatim}
(set-acceleration numerator denominator)
\end{verbatim}}\usageupspace

Accelerates the mouse cursor movement by a ratio of numerator/denominator
(two numbers).

\ITEMa{set-colormap-focus}{sets the window whose colormap is installed}

{\usagefont\begin{verbatim}
(set-colormap-focus [window])
\end{verbatim}}\usageupspace

Installs the colormap of the current window (or the given \verb"window").
Once a window has the {\em colormap focus}, if the
client changes its colormap, the new colormap is automatically installed by
{\GWM}.

If the window has no declared colormap, the default colormap (the colormap
of the root window) is installed instead.

If the argument is \verb"()", the default colormap for the screen is
re-installed.

\ITEMa{set-focus}{sets input focus on a window}
        
{\usagefont\begin{verbatim}
(set-focus [window])
\end{verbatim}}\usageupspace

Sets the focus to the current window, or to the given \verb"window". The
keyboard input will then go to the client of this window, regardless of the
pointer position. If window is (), the focus is reset to {\bf PointerRoot}
mode, i.e., the focus is always on the window under the pointer.

If the client of the current window follows the ICCC
\verb"WM_TAKE_FOCUS" protocol, {\GWM} does not try to set the focus to the
window, it just sends the \verb"WM_TAKE_FOCUS" message to the client
which should then set the focus itself.

{\bf NOTE:} \verb"set-focus" will not set the focus on a client window that
does not need it, so that \verb"set-focus" on XCLOCK for instance has no
effect. This is the only case where \verb"set-focus" will return \verb"()";
it returns \verb"t" otherwise.

\ITEMb{set-grabs}{unset-grabs}{grabs events occurring in the window}

{\usagefont\begin{verbatim}
(set-grabs event1 event2 ... eventN)
(unset-grabs event1 event2 ... eventN)
\end{verbatim}}\usageupspace

{\tt set-grabs} establishes what is called a {\bf passive grab} on a button
or a key on the current window, i.e.; all events matching the given events
will be transmitted to the window itself, even if they have occurred on a
bar, plug, or client window of the window.

{\tt unset-grabs} removes events from the list of grabbed events. They do
not exit an existing active grab.

The {\tt set-grabs} call is used when decorating a window on the
\see{grabs} list.

Example:{\exemplefont\upspace\begin{verbatim}
        (set-grabs (button 3 alone) 
                   (buttonpress any with-alt)
                   (key (key-make "Delete") any))
        (unset-grabs (key any any))
\end{verbatim}}


\ITEMa{set-icon-sizes}{sets desired icon sizes}
        
{\usagefont\begin{verbatim}
(set-icon-sizes min-width min-height
                max-width max-height
                width-inc height-inc)
\end{verbatim}}\usageupspace

This function sets a property of name \verb"WM_ICON_SIZE" on the root window
which tells the clients which are the desirable size for their icon pixmaps
or icon windows, if they define any, as returned by 
\see{window-icon-pixmap} and \see{window-icon-window}.

\ITEMa{set-key-binding}{redefines keyboard for all applications}
        
{\usagefont\begin{verbatim}
(set-key-binding keycode keysym 
                 [keysym1 [keysym2 ... [keysymN]]])
\end{verbatim}}\usageupspace

This rebinds the keys on the server's keyboard. When the key of keycode
\verb"keycode" is pressed, it will be decoded as the keysym
\verb"keysym" if pressed alone, \verb"keysym1" if pressed with
\verb"modifier1",...  \verb"keysymN" if pressed with modifier {\tt N}
(Modifiers are shift, control, lock, meta, etc.)

{\bf WARNING:} The storage used by this function is never released.

{\bf Note:} To obtain the list of current bindings of your server, use the
{\bf xprkbd} X11 utility.

\ITEMa{set-screen-saver}{sets screen-saver parameters}
        
{\usagefont\begin{verbatim}
(set-screen-saver timeout interval 
                  prefer-blanking allow-exposures)
\end{verbatim}}\usageupspace

Sets the way the screen-saver operates:

\begin{description}

\item[Timeout] is the time in seconds of no input after which the screen blanks
                (0 means no screen saver, -1 = restore default)
\item[Interval] is the interval in seconds between random motion of the background
                pattern (0 disables motion)
\item[Prefer-blanking] makes the screen go blank if 1
\item[Allow-exposures] if 0 means that the screen saver operation should not
                generate exposures, even if this implies it cannot operate.
\end{description}

\ITEMa{set-subwindow-colormap-focus}{installs the colormap of a subwindow}

{\usagefont\begin{verbatim}
(set-subwindow-colormap-focus [number])
\end{verbatim}}\usageupspace

If the current window currently has the colormap focus, as set
by the \see{set-colormap-focus} function, and the client has set a
\verb"WM_COLORMAP_WINDOWS" property on its window which tells 
{\GWM} to install
the colormaps of its subwindows, calling \verb"set-subwindow-colormap-focus"
without an argument installs the next different colormap in the list of
subwindows whose colormaps must be installed.

If a numeric argument is given, it is taken as an offset in the list of
subwindows (modulo the size of the list), and the corresponding window's
colormap is installed. The main window is always at offset zero.

Calling \see{set-colormap-focus} on the window resets the current offset
to zero, for susequent calls to \verb"set-subwindow-colormap-focus".


\ITEMa{set-threshold}{sets mouse acceleration threshold}
        
{\usagefont\begin{verbatim}
(set-threshold number)
\end{verbatim}}\usageupspace

Sets the minimum pointer movement in pixels before acceleration takes place
(See \see{mouse-acceleration}). The \verb"number" argument must not be 0.

\ITEMc{set}{setq}{:}{variable assignement}
        
{\usagefont\begin{verbatim}
(setq atom value)
(set object value)
\end{verbatim}}\usageupspace

This is the assignment function of all Lisp dialects. In the first form,
the first argument is not evaluated, in the second it is.
(Thus allowing non-standard atom names to be set via the atom constructor
\verb"atom", as in \verb|(set (atom "foo bar") 1)| ). Both forms evaluate 
their second
argument and sets the value of the first argument to the resulting
value. Setting active values
doesn't modify their value, but calls a predefined function on the value.

\verb":" is just a synonym for \verb"setq".

Example:{\exemplefont\upspace\begin{verbatim}
        (setq b 'c)
        (setq a (+ 1 2))
        (set  b 4)
\end{verbatim}}
yields a = 3 and c = 4.

\ITEMa{set-x-property}{sets an X property on a client window}
        
{\usagefont\begin{verbatim}
(set-x-property property-name value)
\end{verbatim}}\usageupspace

Sets the X11 property of name \verb"property-name" on  the current client
window.  The value currently must be a STRING (or atom) or an INTEGER.  For
instance, \verb|(window-name "foo")| is equivalent to 
\verb|(set-x-property "WM_NAME" "foo")|

This is the recommended way for communicating between {\GWM} and an X 
application.

\ITEMa{sort}{sorts a list in place\hfill{\bf EXPERT}}

{\usagefont\begin{verbatim}
(sort list comparison-function)
\end{verbatim}}\usageupspace

This function sorts (puts in ascending order)
in place the \verb"list" argument using the ``quicksort''
algorithm with the user-provided comparison function.
This function is called on pairs of elements and should return
\verb"-1", \verb"0" or \verb"1" if its first argument is lower than, equal
to, or greater than the second.

This function is flagged as ``{\em Expert}\,'' as it physically modifies the
list. For instance, to obtain a list of windows sorted by names, do:

{\exemplefont\begin{verbatim}
        (sort (list-of-windows) (lambda (w1 w2)
                (compare (with (window w1) window-name)
                         (with (window w2) window-name))))
\end{verbatim}}

\ITEMa{stack-print-level}{number of stack frames printed on error}

\usagetyped{Numeric variable}{number}

On error, {\WOOL} prints a stack dump. The number of stack frames printed
is given by the value of this variable. Setting it to a negative number 
puts no limits on the depth of the dump.


\ITEMa{state-make}{makes a state of a fsm}
        
{\usagefont\begin{verbatim}
(state-make transition1 transition2 ... transitionN)
\end{verbatim}}\usageupspace

Makes a state of a fsm (see \verb"fsm-make" and \verb"on" functions) 
composed of the
transitions given in arguments. Returns the constructed state which can be
affected to a name via \verb"setq" to be used as the destination state in
transitions.

If an argument is itself a state, the new state will {\bf include} all the
transitions in the argument state.

\ITEMa{sublist}{extracts a sub-list out of a list}
        
{\usagefont\begin{verbatim}
(sublist from to list)
\end{verbatim}}\usageupspace

Returns the sublist starting at the from-th element and ending at the to-th
element of list \verb"list" (both from and to are numbers). \verb"from" is
inclusive and \verb"to" is exclusive, and if they are greater than the size
of the list, the elements are set to (). Elements are numbered starting at
0.

\exemples{Examples:\upspace}{
                (sublist 2 4 '(1 2 3 4))&                =={\tt >}  (3 4) \\
                (sublist 5 9 ())        &       =={\tt >}       (() () () ()) \\
                (sublist 10 -8 '(1 2)) &        =={\tt >} ()\\
}

\ITEMa{t}{the logical ``true'' value}
        
{\usagefont\begin{verbatim}
t
\end{verbatim}}\usageupspace

The ``true'' value, evaluates to itself.

\ITEMb{tag}{exit}{non-local goto}
        
{\usagefont\begin{verbatim}
(tag tag-name inst1 inst2 ... instN)
(exit tag-name inst1 inst2 ... instN)
\end{verbatim}}\usageupspace

The pair of functions tag/exit implements a non-local goto.  When \verb"tag"
is called, the non-evaluated tag-name becomes the label of the goto. The
instructions are then evaluated in \verb"progn" fashion. If a call to exit
with the same tag-name (non-evaluated) is made during these instructions, the
evaluation of the tag instructions is aborted and tag returns the
evaluation of the instructions of the exit call, evaluated like progn.

If \verb"exit" makes the flow of control exit from a
\verb"with" local variable declarations construct, the previous variable
values are restored.

{\bf WARNING:} Do not, when in a file being loaded by \see{load} do an
\verb"exit" with a \verb"tag" set outside the load call --- This breaks
{\GWM} in the current version.

\ITEMa{tile}{background pixmap}
        
\usagetyped{Variable}{pixmap}

The value (pixmap) of this global variable is used by the constructors of all
wobs to set their background pixmap. For now, it is only used in bars and
screens.

\ITEMa{together}{combines keyboard modifiers}
        
{\usagefont\begin{verbatim}
(together modifier1 modifier2 ... modifierN)
\end{verbatim}}\usageupspace

Used to indicate that the modifiers must be pressed simultaneously, 

{\exemplefont\begin{verbatim}
        (button 1 (together with-shift with-alt))
\end{verbatim}}

\ITEMb{trace}{trace-level}{traces {\WOOL} function calls}
        
\usagetyped{Active values}{\ }

This is a primitive debugging tool. When the trace value is set to a non-null
number (or \verb"t") every call to any {\WOOL} function is printed,
with the arguments and return value.  If set to an expression, this expression
will be evaluated before and after each list evaluation (Setting
\verb"trace" to \verb"1" instead of \verb"t" re-enables the evaluation of
the previous expression).

The trace level variable holds the
current indentation (stack depth) of the calls.  You might reset it to 0 if
you re-enable the tracing after disabling it at a non-0 level.

{\bf NOTE:} This is a primitive debugging tool, others will be added
in the future.

\ITEMa{trigger-error}{triggers a {\WOOL} error}
        
{\usagefont\begin{verbatim}
(trigger-error exprs...)
\end{verbatim}}\usageupspace

This will trigger an error, returning instantly to the toplevel (unless trapped
by a \seeref{error-occurred}). It will issue an error message, 
print all the given arguments \verb"exprs...", and append a newline.

\ITEMa{type}{type of a {\WOOL} object}
        
{\usagefont\begin{verbatim}
(type object)
\end{verbatim}}\usageupspace

Returns the {\WOOL} type of the object as an atom. Current types are:

\desctable{Atom}{Description}{
active & {\WOOL} active value atom \\
atom & {\WOOL} normal atom \\
bar & bar descriptor \\
client & window descriptor \\
collection & syntax tree node \\
cursor & X cursor \\
event & X event \\
fsm & finite state machine \\
fsm-state & state of a fsm \\
subr & built-in function \\
fsubr & non-evaluating built-in function \\
expr & user-made function (defun) \\
fexpr & non-evaluating user function (defunq) \\
label & active-label \\
list & {\WOOL} list \\
menu & menu used in pop-ups \\
number & number (used for fonts, wobs, colors,...) \\
pixmap & X pixmap \\
plug & plug descriptor \\
pointer & atom pointing to a memory location \\
quoted-expr & quoted expression \\
state-arc & transition arc of a fsm state \\
string & character strings \\
}

\ITEMa{unbind}{undefines a symbol}
        
{\usagefont\begin{verbatim}
(unbind atom)
\end{verbatim}}\usageupspace

Undefine the (evaluated) atom in argument, so that a \verb"boundp" on it will
return nil.

\ITEMa{ungrab-server}{releases grab on the X server}
        
{\usagefont\begin{verbatim}
(ungrab-server [wob])
\end{verbatim}}\usageupspace

Ungrabs the server, allowing other client requests to be processed.
If an argument is given, ungrabs the server only if the argument was the
last wob to grab the server, otherwise does nothing. With no argument, 
unconditionally ungrab the server (keyboard and pointer).

\ITEMa{ungrab-server-and-replay-event}{releases grab on the X server and
replay grabbing event}

{\usagefont\begin{verbatim}
(ungrab-server-and-replay-event flag)
\end{verbatim}}\usageupspace

When the X server has been grabbed by a passive grab (a grab set on a
window by the \verb"grabs" context variable or by the \verb"set-grabs"
function), you can release the grab and make the X server replay the event
as if the grab didn't occur by calling this function.

You must set the \verb"flag" parameter to \verb"()" if the grab was on a
mouse button, and to a non-nil object if the grab was on a key.

This call is useful in {\em click to type} window managers, to re-send the
event which changed the current active window to the client window.

{\bf Note:} An event can only be replayed if it has been grabbed on a
{\bf replayable} event \seesnp{replayable-event}.

\ITEMa{unmap-window}{unmaps (make invisible) a window}
        
{\usagefont\begin{verbatim}
(unmap-window [window])
\end{verbatim}}\usageupspace

Unmaps (makes invisible) the window (or the current one).

\ITEMa{unpop-menu}{makes a popped menu disappear}
        
{\usagefont\begin{verbatim}
(unpop-menu [menu])
\end{verbatim}}\usageupspace

Removes the grab set by the menu (or menu affected to the current wob if no
argument is given) and unmaps it. The {\tt menu} argument can be either a
menu (as returned by the \verb"menu-make" function) or a wob (as provided by
the \verb"wob" active-value). This function synchronize {\GWM} with the
server, so the menu is guaranteed to be unmapped when it returns.

{\bf Warning:} The current-wob is not changed by this function. The wob
which triggered the menu can be accessed as the parent of the menu.

\ITEMa{user-event}{events internal to {\GWM}}
        
{\usagefont\begin{verbatim}
(user-event atom)
\end{verbatim}}\usageupspace

This function is used in transitions to match user-events of the given
(evaluated) atom (see \verb"send-user-event").

Example:
{\exemplefont\upspace\begin{verbatim}
        (on (user-event 'get-focus) (set-pixmap focus-pattern))
\end{verbatim}}

\ITEMa{warp-pointer}{warps the mouse pointer to a location\hfill{\bf EXPERT}}

{\usagefont\begin{verbatim}
(warp-pointer x y [window-relative-to])
\end{verbatim}}\usageupspace

Warps (sets) the mouse pointer to the position \verb"(x,y)", relative to
its current position if no third argument is given, or in the
coordinates of the given \verb"window-relative-to" if given.
For instance, to warp the pointer to the next screen at the same relative
location say:

{\exemplefont\begin{verbatim}
        (setq coordinates (current-mouse-position))
        (setq screen (% (+ (# 3 coordinates) 1) screen-count))
        (warp-pointer (# 0 coordinates) (# 1 coordinates) root-window)
\end{verbatim}}

{\bf WARNING:} Use this function just like you would use ``goto'' in
normal programming: never. \verb"warp-pointer" will ruin a window management policy
just as easily as  ``goto''s will ruin a program.

\ITEMa{while}{while loop}
        
{\usagefont\begin{verbatim}
(while condition inst1 inst2 ... instN)
\end{verbatim}}\usageupspace

Executes the N instructions in sequence until condition becomes ().  Returns
always ().      

\ITEMa{window}{current window id}
        
\usagetyped{Active value}{window id}

Returns or sets the descriptor of the current window as a {\WOOL} number. The
current window is the default for all window functions. Setting window
to a value sets also the current wob to this window.

{\bf WARNING:} A \verb"(with (window w)...)" call will modify the value of
the current wob:
{\exemplefont\begin{verbatim}
                                         ; window is A, wob is B
        (with (window C) (foo))          ; in foo, window is C, wob is C
                                         ; window is A, wob is A
\end{verbatim}}
So, if you want the previous call not to modify \verb"wob", do a 
\verb"(with (wob w)...)" call instead. The same remark holds for the 
\verb"screen" active value: a \verb"(with (screen s)...)" will set the value
of {\tt window} and {\tt wob} to the value of {\tt screen} before the 
{\tt with} call.


\ITEMa{window-client-class}{client application class}
        
\usagetyped{Active value}{string}

Returns a string containing the class of the client owning the window,
e.g., {\bf XTerm} for an xterm window.

\ITEMd{window-client-height}{window-client-width}{window-client-x}{window-client-y}{inner window geometry in pixels}
        
\usagetyped{Active value}{number}

Returns the dimensions in pixels of the client window and its position
inside the {\GWM} window frame.

\ITEMa{window-client-name}{client application name}
        
\usagetyped{Active value}{string}

Returns a string containing the name of the client owning the window, e.g.,
{\bf xterm} for an xterm window.

\ITEMa{window-group}{manages groups of windows}
        
\usagetyped{Active value}{window id}

In X11, windows can be grouped with each window in the group bearing a
reference to a distinguished window, the {\em group leader}. {\GWM}
maintains such groups as a list of windows, the group leader being the first
one in the list. This active value returns () if the window does not belong
to a group, otherwise it returns 
the list of windows in the group.

The user can himself define groups of windows by setting the
\verb"window-group" active value to a window id, which is the group leader
to which the window should be grouped. The entire group (list) can also be
passed as argument, in which case the first element is taken as
the window to be grouped to.

To remove a window from a group, just assign \verb"()" to
\verb"window-group" for this window, it is removed from the group it
was in. If the window was a group leader, the group is broken and all
the windows in it are ungrouped.

{\bf Note:}  A window can only belong to one group.

\ITEMa{window-icon}{icon associated to window}
        
\usagetyped{Active value}{window id --- not settable}

Returns the icon associated with the current window. If the current window
is already an icon, returns the current window itself.

{\bf Note:} When {\GWM} decorates a window, it caches the {\WOOL}
description given for the icon, but does not create it. The icon is
physically created, (and its \verb"opening" field evaluated) on the first
access to the \verb"window-icon" active value or the first call to the
\see{iconify-window} function.

\ITEMa{window-icon-name}{name of the icon}
        
\usagetyped{Active value}{string --- not settable}

Returns the name that the client of the current window has given to its icon.

\ITEMa{window-icon-pixmap}{pixmap to be used in icon}
        
\usagetyped{Active value}{pixmap --- not settable}

Returns the pixmap given as a hint by the current client to be used in its
icon, or () if no pixmap was specified. The bitmap specified by the
application is used to construct the returned pixmap by painting the unset
pixels with the \verb"background" color and the set pixels with the
\verb"foreground" color on the invocation of this function for each window.
Thus each time it is called a new pixmap is created. It is highly
recommended that you store the returned value instead of re-calling the
function another time.

Use it as the \verb"plug" argument of the \verb"window-make" function after
making a plug via \verb"plug-make".

\ITEMa{window-icon-pixmap-change}{pixmap to be used in icon has changed}
        
\usagetyped{Constant}{event}

When an application changes its pixmap to be used as its icon, this event is
generated. You should then use the \verb"window-icon-pixmap" function to
retrieve it if your icon style supports it.

\ITEMa{window-icon-window}{window to be used as icon}
        
\usagetyped{Active value}{window id --- not settable}

Returns a descriptor (number) of the window provided by the current client
to be used as its icon, or () otherwise. Should {\em only} be used as the
\verb"plug" argument of the \verb"window-make" function.

\ITEMa{window-is-mapped}{tells if window is visible}
        
\usagetyped{Active value}{boolean --- not settable}

If the current window is mapped (visible) returns it, () otherwise.

\ITEMa{window-is-shaped}{tells if window has a non-rectangular shape}
        
\usagetyped{Active value}{boolean --- not settable}

If the current client window has a non-rectangular outline (on servers
supporting the {\bf shape} X11 extension), returns \verb"t", \verb"()"
otherwise.

\ITEMa{window-is-transient-for}{tells if window is transient}
        
\usagetyped{Active value}{window --- not settable}

If not (), the window is transient for another window, and thus you might
decide not to decorate it too much. (The window it is transient for is
returned.)

\ITEMa{window-machine-name}{name of host on which the client is running}
        
\usagetyped{Active value}{string --- not settable}

Returns the string containing the name of the machine on which the client
owning the window executes (Defaults to \verb|"machine"| if not set.)

\ITEMa{window-make}{makes a template to decorate a window with}
        
{\usagefont\begin{verbatim}
(window-make titlebar leftbar rightbar basebar plug)
\end{verbatim}}\usageupspace

\centerline{\texpsfig{window.id}{211}{154}}

Returns a description of a {\GWM} window to decorate a newly created X
window. This is also used to describe the associated icon and the screen.
The four bars are the ones that frame the client and are evaluated
another time when the window is physically created. This allows you to give
expressions for bars (quoted to evade the first evaluation of arguments of
the \verb"window-make" function itself) which evaluates to a bar on the
realisation of the wob.  Any bar can be set to (), indicating that that no
corresponding bar should be created.

The fifth argument \verb"plug" is only used when describing an icon, 
to be used as the central window
around which the bars will be framed. You can give a plug (or an expression
which when evaluated gives a plug) or the value of
\verb"window-icon-window" for the window. 
If set to (), the icon has
the dimension of the longest side bar, or if they are also set to (), the
dimension of the longest top or base bar.

\centerline{\texpsfig{icons.id}{236}{91}}

\context{
     fsm &  the fsm of the window \\
     borderwidth & the width of its border \\
     borderpixel & the color of its border \\
     bordertile &  the pixmap tiling its border \\
     inner-borderwidth & the border width of the client window \\
     menu &  the default menu associated to the window \\
     cursor & the shape of the cursor when in the window 
 (in fact in the border, which is the only visible part) \\
     property &  the initial value of the property field \\
     grabs & events grabbed from all sons of the window\\
     opening & {\WOOL} code evaluated on the creation of the window\\
     closing & {\WOOL} code evaluated on the destruction of the window\\
}

\verb"grabs" is a list of \verb"button", \verb"buttonpress", \verb"key" or
\verb"keypress"  events which will be ``grabbed'' from the client window to
be sent to the {\GWM} window. This means that the event will be sent
directly to the {\GWM} window and {\bf not} to the client window.  For
instance, to implement a ``uwm'' move style (moving a window on
Alternate/right button anywhere in the window), the grab list should include
a \verb"(buttonpress 3 with-alt)", and the fsm of the window should have a
\verb"(on (buttonpress 3 with-alt) (move-window))" transition.

Events declared in the \verb"grabs" list are trapped on the whole surface of
the window, including the bars and the client window, and redirected to the
main window's fsm.

\verb"opening" and \verb"closing" are two Lisp expressions that are
evaluated when the window (or icon) is created and destroyed respectively,
just before being mapped or unmapped. This is the right place to position or
resize the window before it appears. For the screen, opening is evaluated
once all windows already on screen have been framed and closing is evaluated
when leaving {\GWM}.

When used for screen description, no arguments are used, only the context
values for grabs, opening, closing, fsm, menu, cursor, and property context
variables, with the additional context value:

\context{
            tile        & tiling the screen with a pixmap (if set
                                to a pixmap) or defining the screen color
                                (if set to a color). () means do not change
                                the screen background. \\
}

\ITEMa{window-name}{name of the window}
        
\usagetyped{Active value}{string}

Returns the string containing the name of the window, as set by the client
owning it. Note that it is a transient property, and the event
\verb"name-change" is issued when it is changed by the client.  

When \verb"window-name" is set, the \verb|WM_NAME| X property on the
client window is updated accordingly, resulting in the X \verb"name-change"
event to be sent to the window by the X server.

{\bf Note:} All dots in the name are converted to underscores, so that you
can use the value of \verb"window-name" safely as an name for the X resource
manager.

\ITEMb{window-property}{wob-property}{{\WOOL} property associated to a wob}
        
\usagetype{Active value}

To each wob is associated a property, which is any Lisp object given by the
user. These active values return or set the stored property for the
current wob or window. When
creating a wob, the property is taken as the current value of the global
variable \verb"property".

\ITEMa{window-size}{client window size specified in resize increments}
        
\usagetyped{Active value}{list of two numbers}

Returns the size of the window expressed as a list of two numbers,
multiples of the minimal size (e.g. character positions for xterm).  When
set, the window is resized accordingly. This is the size of the inner
client window, to specify the outer dimensions, use \verb"resize-window".

\ITEMa{window-starts-iconic}{state in which window must first appear}
        
\usagetyped{Active value}{boolean}

If not (), the window appears as an icon on its first mapping or the
first time it is decorated by {\GWM}. This ``hint'' can thus be set either
by the application or the window manager.

\ITEMb{window-status}{wob-status}{state of the window}
        
\usagetyped{Active value}{atom --- not settable}

Returns the type of the current wob or window as an ATOM. Possible types
are:

\desctable{Atom}{type of wob}{
window & main window around a client \\
icon & icon of a window \\
menu    &       menu created with \verb"menu-make" \\
root            &the root window \\
bar     &       a bar \\
plug &          a plug \\
}

Thus, to check if the current window is an icon, just say:
{\exemplefont\begin{verbatim} 
        (if (= window-status 'icon) ...)
\end{verbatim}}

\ITEMb{window-user-set-position}{window-user-set-size}{tells if user
explicitly specified the geometry}
        
\usagetyped{Active value}{boolean --- not settable}

Return \verb"t" if the position or size of the current window was set
explicitly by the user at creation time via command line switches, ()
otherwise. If called on an icon, tells if the user defined the initial icon
position.

\ITEMb{window-program-set-position}{window-program-set-size}{tells if program
explicitly specified the geometry}
        
\usagetyped{Active value}{boolean --- not settable}

Return \verb"t" if the position or size of the current window was set
by default by the program.

\ITEMa{window-was-on-screen}{tells if window was already on screen}
        
\usagetyped{Active value}{boolean --- not settable}

Returns \verb"t" if the window was already on the screen before {\GWM}.  You may test
its value to do certain actions only on newly created windows (if ()), such
as interactive placement.

\ITEMd{window-width}{window-height}{wob-height}{wob-width}{window dimensions in pixels}
        
\usagetyped{Active value}{number --- not settable}

These functions return the size of the current window (with decoration) or
wob in pixels.  These do not include the width of the border, following the
X11 conventions.

\ITEMa{window-window}{main window associated with an icon}
        
\usagetyped{Active value}{window id --- not settable}

Returns the window associated with the current window if it is an icon. If
the current window is not an icon, returns the current window itself.


\ITEMa{window-wm-state-icon}{declares user icon for WM\_STATE}
        
\usagetyped{Active value}{window id}

{\GWM} manages automatically the \verb"WM_STATE" property on client windows.
However, if you implement in a {\WOOL} package your own icons which are not
the {\GWM} ones, which are managed by the 
\verb"window-icon" and \verb"iconify-window"
primitives, for instance by creating a window via the \verb"place-menu"
primitive, you need to declare which window must logically be
considered as the icon for your window. This is done by setting the
\verb"window-wm-state-icon" on the window to the icon.

Once you have declared that a window has a user-managed icon, {\GWM} no
longer updates the \verb"WM_STATE" property, so you should call
\verb"window-wm-state-update" each time you change the state of the window

The state of the window can be:

\begin{tabular}{|l|c|c|}
\hline
\vbox{\hbox{window}\hbox{icon}} & mapped & unmapped \\
\hline
mapped & {\bf normal} & {\bf iconic} \\
\hline
unmapped & {\bf normal} & {\bf withdrawn} \\
\hline
\end{tabular}

This state information is used by session managers and other window
managers, for instance {\GWM} itself will re-iconify the windows that were
previouly iconified on a restart.

\ITEMa{window-wm-state-update}{updates WM\_STATE property for windows with user icon}

{\usagefont\begin{verbatim}
(window-wm-state-update)
\end{verbatim}}\usageupspace

Once you have declared that a window has a user-managed icon, you should
update the \verb"WM_STATE" property by calling this function with the
\verb"window" global variable set to the managed window each time the state
of the window changes.

\ITEMb{window-x}{window-y}{position of upper-left corner of window in root coordinates}
        
\usagetyped{Active value}{number --- not settable}

These functions return the coordinates of the upper-left corner of the
current window, including its decoration, in the root window. 

\ITEMb{with}{with-eval}{local variable declaration}
        
{\usagefont\begin{verbatim}
(with (var1 value1 ... varN valueN) instructions ...)
(with context instructions ...)
(with-eval expression instructions ...)
\end{verbatim}}\usageupspace

This is the construct used to declare and initialize variables local to a
group of instructions. Active-values and functions are handled as expected,
resetting their initial value after the execution of the body of the with.
The values are evaluated sequentially.

A context is a list of variables and associated values that can be re-used
in many \verb"with" functions (see \verb"context-save").

\verb"with-eval" first evaluates the expression and then uses it as a
context, so that the two following calls are equivalent:

{\exemplefont\begin{verbatim}
        (with (a 1 b 2) c)
        (with (+ '(a 1) '(b 2)) c)
\end{verbatim}}

Due to the structure of the {\WOOL} interpreter, \verb"with" works also with
\verb"active-values" and functions. 
For example the following call can be made to
move the window ``my-window'' in the upper left corner of the screen.

{\exemplefont\begin{verbatim}
        (with (window my-window move-ul (lambda () (move-window 0 0)))
                (move-ul))
\end{verbatim}} 

Example: {\exemplefont\upspace\begin{verbatim}
        (setq bluegreen '(foreground green background blue))
        (with bluegreen (pixmap-make "Bull"))
        (with (a 2 b (+ a 1)) (print b))        ==>     3
\end{verbatim}}

\ITEMf{with-shift}
        {with-control}
        {with-alt}
        {with-lock}
        {with-modifier-N}
        {with-button-N}{modifier states}
        
\usagetyped{Constants}{number}

These numerical values describe what was in a ``down'' state (i.e., pressed)
when a key or button event was sent. You can combine them with the
\verb"together" function, for instance if you want shift {\bf and} control 
pressed.

{\bf N} can takes values 2 to 5 for modifiers and 1 to 5 for buttons, e.g.,
the \verb"with-modifier-5" and \verb"with-button-1" variables are defined.

\ITEMa{wob}{current wob}
        
\usagetyped{Active value}{wob id}

Returns the descriptor of the current wob (i.e., the wob which received the
event being processed) as a {\WOOL} number.  The current wob is the default for
all wob functions.  When set, the current wob is now the wob argument.

\ITEMa{wob-at-coords}{wob containing coordinates}

{\usagefont\begin{verbatim}
(wob-at-coords x y)
\end{verbatim}}\usageupspace

Returns the wob including the coordinates relative to the root. Returns () if
coordinates were off-screen.

\ITEMa{wob-background}{(solid) background color of wob}
        
\usagetyped{Active value}{color}

Returns or sets the (solid) background color of the wob. If set, discards the
tile of the wob (if there was one) to paint its background with a solid color.
Works only with bars and screens.

\ITEMa{wob-borderpixel}{color of border}
        
\usagetyped{Active value}{color}

Get or set the color of the border of the current wob.

\ITEMa{wob-borderwidth}{width of the border of a wob}
        
\usagetyped{Active value}{number}

Gets or sets the width in pixels of the border of the current wob.  If it is in
a composite wob, the father is then resized.

\ITEMa{wob-cursor}{cursor displayed when pointer is in a wob}

\usagetyped{Active value}{cursor}

This active value allows the user to get or modify the mouse cursor which
is displayed when the pointer is in the current wob.

\ITEMa{wob-fsm}{gets or sets the fsm associated with current wob}
        
\usagetyped{Active value}{fsm}

This active value allows the user to get or  modify the fsm associated with
a wob.  If you set the fsm of a wob, it is placed in the initial state.
This function is intended for debugging purposes, and its use should be
avoided in normal operation. Try using multiple-state fsms instead of
changing the fsm.

\ITEMa{wob-invert}{quick and dirty inversion of a wob}
        
{\usagefont\begin{verbatim}
(wob-invert)
\end{verbatim}}\usageupspace

Inverts the colors over the surface of the
current wob. This is just drawn on top of everything
else, and does not affect permanently the wob. This should only be used after
calling \verb"grab-server" and then \verb"process-exposes"
(Note that \verb"pop-menu" does that for you).
This is a ``lightweight'' function to be used on transient objects.
To invert a wob in a cleaner way, use \see{wob-tile}.

The wob is inverted by using the \verb"invert-color" color (screen relative).
This color should be choosen to be the {\bf bitwise-xor}ing of two colors
that will be inverted by this functions, the other colors being affected
in an unpredictable way.

\ITEMa{wob-menu}{gets or sets the menu associated with current wob}
        
\usagetyped{Active value}{menu}

This active value allows the user to get or 
modify the menu associated with a wob.

\ITEMa{wob-parent}{finds the parent of a wob}
        
\usagetyped{Active value}{wob id --- not settable}

Returns the parent of the current wob.  Note that the parent of a pop-up is
the wob which called the \verb"pop-menu" function. The parent of windows and
icons is the root window itself, and the parent of a root window is {\bf nil}.

\ITEMb{wob-tile}{wob-pixmap}{graphic displayed in a wob}
        
\usagetyped{Active value}{pixmap}

Returns or sets what is the current wob's pixmap background. For the screen
itself or a bar, this is its background tile; for a plug this is
the pixmap around which the plug is built. If you change the size of the
\verb"wob-tile" for a plug, it is automatically resized (but not a bar
or a screen). \verb"wob-pixmap" is just a synonym for \verb"wob-tile"
for backward compatibilty purposes.

\ITEMb{wob-x}{wob-y}{absolute screen position in pixels of current wob}

\usagetyped{Active values}{wob id --- not settable}

Returns the position of the top left corner of the wob, including its border
as absolute pixel coordinates in its screen.
